\section{The triangulated Rosetta stone}\label{t_structure_subs}

As in Section \refbf{section_homo_FS}, let us fix throughout this section a triangulated category $\cD$ with shift functor $\Sigma\colon \cD \xto{\simeq} \cD$.
\begin{definition}\label{def:teestru}
Recall that a \emph{$t$-structure} in $\cD$ is a pair $\t = (\cD^{\leq  0}, \cD^{\geq 0}  )$ of full sub-categories of $\cD$ that satisfy the following properties, where $\cD^{\leq n} \coloneqq \Sigma^{-n}\cD^{\leq 0}$ and $\cD^{\geq n} \coloneqq \Sigma^{-n} \cD^{\geq 0} $, for any $n\in\Z$:
\begin{enumerate}[label=t\arabic*)]
\item\label{tee:fst} $\cD(X,Y)=0$ for any $X \in \cD^{\leq  0}$ and $Y \in  \cD^{\geq  1} $;
\item\label{tee:snd} $\cD^{\leq-1}\subseteq \cD^{\leq  0}$ and $ \cD^{\geq 1} \subseteq  \cD^{\geq  0} $;
\item\label{tee:trd} for any $X \in \cD$ there is a distinguished triangle 
\[
X^{\leq0} \to X \to X^{\geq1}\to \Sigma X^{\leq0},
\] 
with $X^{\leq0} \in \cD^{\leq 0}$ and $X^{\geq1}\in  \cD^{\geq 1} $.
\end{enumerate}
\end{definition}
Given a $t$-structure $\t = (\cD^{\leq  0}, \cD^{\geq 0} )$ in $\cD$, one obtains two functors
\[
\tau^{\leq0}\colon \cD\to \cD^{\leq  0} \quad\text{and}\quad \tau^{\geq1}\colon \cD\to  \cD^{\geq  1} ,
\]
that are respectively the right adjoint to the inclusion $\cD^{\leq  0}\to \cD$ and the left adjoint to the inclusion $ \cD^{\geq  1} \to \cD$.


\begin{notat}
For an object $X\in\cD$ we will generally write $X^{\leq0}$ for $\tau^{\leq0}X$ and $X^{\geq1}$ for $\tau^{\geq1}X$. Furthermore, we will generally denote the unit of the co-reflection $\tau^{\leq0}$ and the co-unit of the reflection $\tau^{\geq1}$ by the following symbols: 
\[
X^{\leq0} \xto{\sigma_X} X \xto{\rho_X} X^{\geq1}.
\]
For any $n\in\Z$, we let $\tau^{\leq n} \coloneqq \Sigma^{-n}\tau^{\leq0}\Sigma^n$ and $\tau^{\geq n} \coloneqq \Sigma^{-n}\tau^{\geq0}\Sigma^n$. We adopt similar notational conventions for these shifted functors.
\end{notat}

%Notice that
%\begin{itemize}
%\item Associating to $X\in\cD$ the triangle of axiom 2 above defines a (unique up to isomorphism) functor
%\[
%\tau_\tee\colon \cD \longrightarrow \text{Tria}(\cD).
%\]
%Moreover, composing $\tau_\tee$ with the projection to the left (resp. right) vertex defines the right (resp. left) adjoints $\tau^{\leq0}, \tau^{\geq 1}$;
%\item Associating to a $t$-structure $\tee$ the functor $\tau_\tee$ defines a functor
%\[
%	\tau_\bullet \colon \ts(\cD)^\opp \longrightarrow \text{Tria}(\cD)^\cD,
%\]
%(where $\ts(\cD)$, the class of all $t$-structures on $\cD$, is thought as a posetal categorY).
%\end{itemize}


\begin{remark}
We can equally define a $t$-structure as a single full additive subcategory  $\t\subseteq \cD$ such that 
\begin{itemize}
	\item $\Sigma\tee \subseteq \tee$;
	\item each object $X\in\cD$ fits into a distinguished triangle $X_{\tee}\to X\to X_{\tee^\perp}\to \Sigma X_{\tee}$ such that $X_\tee\in\tee$, $X_{\tee^\perp}\in\tee^\perp = \{Y\mid \cD(X,Y)=0,\; \forall X\in\tee\}$.
\end{itemize}
This equivalent description of $t$-structures calls $\tee$ an \emph{aisle} and $\tee^\perp$ a \emph{coaisle}.  We will usually blur the distinction between a $t$-structure and its aisle, since the correspondence between the two is obviously bijective under $\cD^{\leq 0} \leftrightarrows \text{aisle}$.
\end{remark}

\subsection{The induced \hfs of a $t$-structure}
Fix a $t$-structure $\t = (\cD^{\leq  0}, \cD^{\geq 0} )$ in $\cD$, and consider the following two classes of morphism 
\begin{align}
\E_\t &  \coloneqq \{\phi\in \cD^\due \mid\tau^{\geq1}\phi \text{ is an iso}\}\notag\\
\M_\t &  \coloneqq \{\psi\in \cD^\due \mid\tau^{\leq0}\psi \text{ is an iso}\}.
\end{align}
This subsection is devoted to the proof of the fact that $\F_\t \coloneqq (\E_\t,\M_\t)$ is a \hfs.

\begin{lemma}[cartesian characterization of $\F_\t$]\label{classes_via_cartesian}
In the above setting, a morphism $(\phi\colon X\to Y)\in\cD^\due$ belongs to $\E_\t$ if and only if the  square
\begin{equation}\label{cartesian?}
\xymatrix{
X^{\leq0}\ar@{}[dr]\ar[r]^{}\ar[d]_{\phi^{\leq0}}&X\ar[d]^{\phi}\\
Y^{\leq0}\ar[r]_{}&Y
}
\end{equation}
is homotopy cartesian. Thus, if $\phi\in\E_\t$, the cone of $\phi$  belongs to $ \cD^{\leq 0} $. Dually, $(\psi\colon X\to Y)\in\cD^\due$ belongs to $\M_\t$ if and only if the square
\begin{equation*}
\xymatrix{
X\ar@{}[dr]\ar[r]^{}\ar[d]_{\psi}&X^{\geq1}\ar[d]^{\psi^{\geq1}}\\
Y\ar[r]_{}&Y^{\geq1}
}
\end{equation*}
is homotopy cartesian. Thus, if $\psi\in\M_\t$, the cone of $\psi$ belongs to $ \cD^{\geq 0} $.
\end{lemma}
\begin{proof}
Suppose first that $\phi\in \E_\t$. 
By \cite[Remark \textbf{1.3.15}]{Neeman}, the square in \eqref{cartesian?} can be completed to a good morphism of triangles
\[
\xymatrix{
X^{\leq0}\ar@{}[dr]\ar[r]^{}\ar[d]_{\phi^{\leq0}}&X\ar[d]^{\phi}\ar[r]&X^{\geq1}\ar[r]\ar@{.>}[d]&\Sigma X\ar[d]\\
Y^{\leq0}\ar[r]_{}&Y\ar[r]&Y^{\geq1}\ar[r]&\Sigma Y
}
\]
while by \cite[\aprop\textbf{1.1.9}]{BBD}, the unique map completing the above square to a morphism of triangles is $\tau^{\geq1}\phi$. Thus, we get that the following candidate triangle is in fact a triangle
\[
X^{\leq0}\oplus\Sigma^{-1}Y^{\geq1}\to X\oplus Y^{\leq0}\to X^{\geq1}\oplus Y\to \Sigma X\oplus Y^{\geq1}.
\]
The above triangle is the direct sum of the following candidate triangles (see \cite[Lemma \textbf{1.2.4}]{Neeman})
\[
\Sigma^{-1}Y^{\geq1}\to 0\to X^{\geq1}\tilde\to Y^{\geq1}\ \ \text{ and }\ \ X^{\leq0}\to X\oplus Y^{\leq0}\to Y\to \Sigma X,
\]
showing that the candidate triangle on the right-hand-side is a distinguished triangle (as it is a summand of a distinguished triangle). The existence of such a triangle means exactly that the square in \eqref{cartesian?}  is homotopy cartesian.

On the other hand, suppose the square in \eqref{cartesian?} is homotopy cartesian. By \cite[Remark \textbf{1.4.5}]{Neeman}, this can be completed to a good morphism of triangles
\[
\xymatrix{
X^{\leq0}\ar@{}[dr]\ar[r]^{}\ar[d]_{\phi^{\leq0}}&X\ar[d]^{\phi}\ar[r]&X^{\geq1}\ar[r]\ar@{.>}[d]|\cong&\Sigma X\ar[d]\\
Y^{\leq0}\ar[r]_{}&Y\ar[r]&Y^{\geq1}\ar[r]&\Sigma Y
}
\]
Invoking again  \cite[\aprop\textbf{1.1.9}]{BBD}, we obtain that $\tau^{\geq1}\phi$ is an iso.
\end{proof}
\begin{lemma}\label{cart_clos}
Consider a homotopy cartesian square 
\[
\xymatrix{
X\ar@{}[dr]|\boxvoid\ar[r]\ar[d]_{\phi}&Y\ar[d]^{\psi}\\
X'\ar[r]&Y'
}
\]
If $\phi^{\geq 0}$ is an isomorphism, then $\psi^{\geq 0}$ is an isomorphism. Dually, if $\psi^{\leq 0}$ is an isomorphism, then $\phi^{\leq 0}$ is an isomorphism. In other words, $\E_\t$ is closed under homotopy pushouts and $\M_\t$ is closed under homotopy pullbacks.
\end{lemma}
\begin{proof}
Suppose first that $\phi^{\geq0}$ is an isomorphism. This means that $ \cD^{\geq  0} (\phi^{\geq 0},B)$ is an isomorphism for any $B\in  \cD^{\geq  0} $ or, equivalently, $\cD(\phi,B)$ is an isomorphism for any $B\in  \cD^{\geq 0} $. We have to show the same property holds for $\psi$. Consider the following morphism of triangles:
\[
\xymatrix{
Z\ar[r]\ar@{=}[d]&X\ar@{}[dr]|\boxvoid\ar[r]\ar[d]_{\phi}&Y\ar[d]^{\psi}\ar[r]&\Sigma Z\ar@{=}[d]\\
Z\ar[r]&X'\ar[r]&Y'\ar[r]&\Sigma Z
}
\]
For any given $B\in  \cD^{\geq 0} $, we obtain a morphism of long exact sequences:
\[
\xymatrix@C=15pt{
\cdots\ar[r]&\cD(\Sigma X',B)\ar[r]\ar[d]|{\cong}&\cD(\Sigma Z,B)\ar[r]\ar@{=}[d]&\cD(Y',B)\ar[r]\ar[d]&\cD(X',B)\ar[r]\ar[d]|{\cong}&\cD(Z,B)\ar[r]\ar@{=}[d]&\cdots\\
\cdots\ar[r]&\cD(\Sigma X,B)\ar[r]&\cD(\Sigma Z,B)\ar[r]&\cD(Y,B)\ar[r]&\cD(X,B)\ar[r]&\cD(Z,B)\ar[r]&\cdots
}
\]
where $\cD(\Sigma\phi,B)$ is an isomorphism because $\cD(\phi,\Sigma^{-1}B)$ is an isomorphism, since $\Sigma^{-1}B\in  \cD^{\geq 1} \subseteq  \cD^{\geq  0} $. Now, by the Five Lemma we obtain that $\cD(\psi,B)$ is an isomorphism for any $B\in  \cD^{\geq  0} $, that is, $\psi^{\geq0}$ is an isomorphism. The proof of the second part of the statement is dual.
\end{proof}
\begin{lemma}\label{all_maps_are_crumbled}
Any morphism in $\cD$ is $\F_\t$-crumbled. 
\end{lemma}
\begin{proof}
Take a map $\phi\colon X\to Y$ in $\cD$, and let us prove that $\phi$ is $\F_\t$-crumbled. Let us start taking a homotopy pullback of the maps $\phi^{\geq1}$ and $\rho_Y$:
\[
\xymatrix{
P\ar@{.>}[r]\ar@{.>}[d]_{\phi_m}\ar@{}[dr]|{\boxvoid}&X^{\geq1}\ar[d]^{\phi^{\geq1}}\\
Y\ar[r]_{\rho_Y}&Y^{\geq1}
}
\]
By Lemma \refbf{classes_via_cartesian}, $\phi_m\in \M_\t$. Consider also the following commutative solid diagram
\[
\xymatrix{
X\ar@{.>}[dr]|{\exists \phi_e}\ar@/_-10pt/[rrd]^{\rho_X}\ar@/_10pt/[rdd]_\phi\\
&P\ar[r]\ar[d]\ar@{}[dr]|{\boxvoid}&X^{\geq1}\ar[d]^{\phi^{\geq1}}\\
&Y\ar[r]_{\rho_Y}&Y^{\geq1}
}
\]
Then there exists a (non-unique, see \cite[p. \textbf{54}]{Neeman}) map $\phi_e\colon X\to p$ that makes the diagram commute. Finally consider the following diagram, where the dotted arrow is obtained completing to a good map of triangles:
\[
\xymatrix{
X^{\leq0}\ar@{.>}[d]\ar[r]^{\sigma_X}&X\ar[d]|{\phi_e}\ar[r]^{\rho_X}&X^{\geq1}\ar@{=}[d]\ar[r]&\Sigma X^{\leq0}\ar[d]\\
Y^{\leq0}\ar@{=}[d]\ar[r]&P\ar[r]\ar[d]|{\phi_m}\ar@{}[dr]|{\boxvoid}&X^{\geq1}\ar[d]^{\phi^{\geq1}}\ar[r]&\Sigma Y^{\leq0}\ar@{=}[d]\\
Y^{\leq0}\ar[r]_{\sigma_Y}&Y\ar[r]_{\rho_Y}&Y^{\geq1}\ar[r]&\Sigma Y^{\leq0}
}
\]
By construction $\phi=\phi_m\phi_e$. It remains to show that $\phi_e\in \E_\t$. By Lemma \refbf{classes_via_cartesian}, we have to verify that the top left square is homotopy cartesian. Indeed, take the following mapping cone, which is distinguished since we took a good morphism of triangles in our construction:
\[
X\oplus Y^{\leq0}\to P\oplus X^{\geq1}\to X^{\geq1}\oplus \Sigma X^{\leq0}\to\Sigma X\oplus \Sigma Y^{\leq 0}.
\]
This triangle is the direct sum of the following two candidate triangles (see \cite[Lemma \textbf{1.2.4}]{Neeman}):
\begin{gather*}
0\to X^{\geq1}\to X^{\geq1}{\to}0,\\
X\oplus Y^{\leq0}\to P\to  \Sigma X^{\leq0}{\to}\Sigma X\oplus\Sigma Y^{\leq0},
\end{gather*}
showing that $X^{\leq0}\to X\oplus Y^{\leq0}\to P \to \Sigma X^{\leq0}$ is distinguished.
\end{proof}




\begin{lemma}
Given $e\in \E_\t$ and $m\in\M_\t$, we have $e \horth  m$. 
\end{lemma}
\begin{proof}
Complete $e$ and $m$ to triangles as follows:
\[
E_0 \xto{e}  E_1 \xto{\alpha_e} C_e \xto{\beta_e} \Sigma E_0\ \ \ M_0 \xto{m}  M_1 \xto{\alpha_m} C_m \xto{\beta_m} \Sigma M_0,
\] 
By Lemma \refbf{classes_via_cartesian}, there are morphisms of triangles, with $\phi=e^{\leq0}$ and $\psi=m^{\geq1}$,
\[
\xymatrix{
X_0\ar@{}[dr]|{\square}\ar[d]_\phi\ar[r]&E_0\ar[d]^e&&M_0\ar@{}[dr]|{\square}\ar[d]_m\ar[r]&Y_0\ar[d]^\psi\\
X_1\ar[d]_{\alpha_e'}\ar[r]&E_1\ar[d]^{\alpha_e}&&M_1\ar[d]_{\alpha_m}\ar[r]&Y_1\ar[d]^{\alpha_m'}\\
C_e\ar[d]_{\beta_e'}\ar@{=}[r]&C_e\ar[d]^{\beta_e}&&C_m\ar[d]_{\beta_m}\ar@{=}[r]&C_m\ar[d]^{\beta_m'}\\
\Sigma X_0\ar[r]&\Sigma E_0&&\Sigma M_0\ar[r]&\Sigma Y_0}
\]
where $X_0, X_1\in  \cD^{\leq 0} $ and $Y_0, Y_1\in  \cD^{\geq 1} $. Using the closure properties of $ \cD^{\leq 0} $ and $ \cD^{\geq 1} $, one can show that $C_e\in  \cD^{\leq 0} $ and $\Sigma^{-1}C_m\in  \cD^{\geq  1} $. Thus, $\cD(C_e,\Sigma^{-1}C_m)=0$ by condition \refbf{def:teestru}.\refbf{tee:fst}, giving us  \refbf{wobbly}.\textsc{ho}1 It remains to verify condition \refbf{wobbly}.\textsc{ho}2, that is, suppose  we have a map $f\colon C_e\to C_m$ whose image in $\cD(E_1,\Sigma M_0)$ is trivial and let us prove that $f=0$. Indeed, we know that $\beta_mf\alpha_e=0$, so also $\beta'_mf\alpha'_e=0$ and thus we can find a morphism of triangles as follows
\[
\xymatrix{
0\ar[r]\ar[d]&X_1\ar@{=}[r]\ar@{.>}[d]|{f_1}&X_1\ar[r]\ar[d]|{f\alpha'_e}&0\ar[d]\\
Y_0\ar[r]&Y_1\ar[r]^{\alpha'_m}&C_m\ar[r]^{\beta'_m}&\Sigma Y_0
}
\]
showing that $f\alpha'_e=\alpha'_m f_1$ for some $f_1\colon X_1\to Y_1$. But $\cD(X_1,Y_1)=0$ by \refbf{def:teestru}.\refbf{tee:fst}, so $f_1=0$, showing that $f\alpha'_e=0$. Hence, we can find a morphism of triangles as follows
\[
\xymatrix{
X_0\ar[r]^{e}\ar[d]&X_1\ar[r]^{\alpha'_e}\ar[d]&C_e\ar[r]^{\beta_e'}\ar[d]|{f}&\Sigma X_0\ar@{.>}[d]^{f_2}\\
\Sigma^{-1}C_m\ar[r]&0\ar[r]&C_m\ar@{=}[r]&C_m
}
\]
showing that $f=f_2\beta_e'$, for some $f_2\colon \Sigma X_0\to C_m$. Now, since $\Sigma X_0\in  \cD^{\leq -1} $ and $C_m\in  \cD^{\geq 0} $, $f_2=0$ and so also $f=0$, as desired.
\end{proof}
\begin{proposition}
The pair of sub categories $\F_\t=(\E_\t,\M_\t)$ defines a \hfs.
\end{proposition}
\begin{proof}
We have already seen that any morphism is $\F_\t$-crumbled and that $\E_\t\subseteq \lhorth{\!\M_\t}$. Let us show the converse inclusion. Indeed, let $(\phi\colon X\to Y)\in \lhorth{\!\M_\t}$ and choose a factorization $\phi=\phi_m\phi_e$ with $\phi_e\in \E_\t$ and $\phi_m\in\M_\t$. By the usual $3\times 3$-lemma in triangulated categories, we can complete the commutative square
\[
\xymatrix{
X\ar[r]^{\phi_e}\ar[d]_{\phi}&p\ar[d]^{\phi_m}\\
Y\ar@{=}[r]&Y
}
\]
to a diagram where all the rows and columns are distinguished triangles, and where everything commutes but the top left square, that anti-commutes:
\[
\xymatrix{
\Sigma^{-1}C_e\ar@{=}[d]\ar[r]&\Sigma^{-1}C_\phi\ar[d]\ar[r]&\Sigma^{-1}C_m\ar[d]\ar[r]&C_e\ar@{=}[d]\\
\Sigma^{-1}C_e\ar[d]\ar[r]&X\ar[d]_\phi\ar[r]^{\phi_e}&P\ar[d]^{\phi_m}\ar[r]&C_e\ar[d]\\
0\ar[r]\ar[d]&Y\ar@{=}[r]\ar[d]&Y\ar[r]\ar[d]&0\ar[d]\\
C_e\ar[r]&C_\phi\ar[r]&C_m\ar[r]&\Sigma C_e}
\]
Now, since $\phi\in \lhorth{\!\M_\t}$, it follows by \refbf{wobbly}\textbf{.}\textsc{ho2}' that the map $C_\phi\to C_m$ in the above diagram is the trivial map. Thus, $\Sigma C_e\cong C_m\oplus \Sigma C_\phi$, in particular $C_m$ is a summand of $\Sigma C_e\in \Sigma  \cD^{\leq  0} = \cD^{\leq -1} $. Hence, $C_m\in  \cD^{\leq -1} \cap  \cD^{\geq 0} =0$, showing that $\phi_m$ is an isomorphism, so that $\phi\cong \phi_e\in \E_\t$. 
\end{proof}

\subsection{$t$-structures are  normal \htth}
We now concentrate on showing how each $t$-structure on $\cD$ naturally induces a \htth and vice-versa; the basic idea is to mimic the proof of \cite[\athm\textbf{3.1.1}]{tstructures} tailoring the argument to the triangulated setting.

\begin{lemma}
$\F_\t=(\E_\t,\M_\t)$ is a normal \htth.
\end{lemma}
\begin{proof}
We have already proved that $\F_\t$ is a \hfs, while the fact that $\E_\t$  and $\M_\t$ are $3$-for-$2$ classes is a trivial consequence of their definition, as they are the pre-image (under $\tau^{\geq1}$ and $\tau^{\leq0}$, respectively) of the class of all isomorphisms, which is a $3$-for-$2$ class. It remains to show that $\F_\t$ is normal. Consider a factorization of a final map $X\to 0$ as follows
\[
X \xto{e} T \xto{m}  0\ \ \text{ with }\ \ e\in \E_\t,\ m\in \M_\t,
\] 
and a triangle of the form $R\to X \xto{e} T\to \Sigma R$. We should prove that the map $(R\to 0)$ belongs to $\E_\t$, that is, that $R\in  \cD^{\leq 0} $. By Lemma \refbf{classes_via_cartesian}, $T\in  \cD^{\geq 1} $. Since $e\in \E_\t$ and using Lemma \refbf{classes_via_cartesian}, we can construct a commutative diagram as follows:
\[
\xymatrix{
X^{\leq0}\ar@{}[dr]|\square\ar[r]\ar[d]&X\ar[r]\ar[d]^{e}&X^{\geq1}\ar[r]\ar[d]|\cong&\Sigma X^{\leq0}\ar[d]\\
T^{\leq0}\ar[r]&T\ar[r]&T^{\geq1}\ar[r]&\Sigma T^{\leq0}}
\]
Since $T\in  \cD^{\geq 1} $, we get $T^{\leq0}=0$ and $T\cong T^{\geq 1}\cong X^{\geq1}$, so the fact that the square on the left-hand-side in the above diagram is homotopy cartesian provides us with a distinguished triangle of the form
\[
X^{\leq0}\to X\to T\to \Sigma X^{\leq0}.
\]
In particular, $R\cong X^{\leq0}\in  \cD^{\leq 0} $ as desired.
\end{proof}
\begin{lemma}\label{htt_induces_t_structure}
For a normal \htth $\F=(\E,\M)$ in $\cD$, $\t_\F \coloneqq (0/\E,\Sigma(\M/0))$ is a $t$-structure.
\end{lemma}
\begin{proof}
We verify the three axioms of a $t$-structure:%, as stated in Subsection \refbf{t_structure_subs}:
\begin{itemize}%[label=t\arabic*]
\item Let $X\in 0/\E$ and $Y\in \M/0$, we have to show that $\cD (X,Y)=0$. Indeed, let $\varphi \colon X\to Y$ and consider the following diagram
\[
\xymatrix{
0\ar[r]\ar[d]&0\ar[d]\\
X\ar[r]^{\varphi}\ar@{=}[d]&Y\ar@{=}[d]\\
X\ar[r]^{\varphi}\ar[d]&Y\ar[d]\\
0\ar[r]&0
}
\]
Notice that $(0\to X)\in \E$. Furthermore, $0\to 0$ is an isomorphism so it belongs to $\M$, as well as $Y\to 0$; since $\M$ is a $2$-for-$3$ class, this means that also $0\to Y$  belongs to $\M$. By condition \refbf{wobbly}\textbf{.}\textsc{ho2}, we get $\varphi=0$.
\item Let $X\in 0/\E$. Reasoning as in verifying \refbf{def:teestru}.\refbf{tee:fst} above, one can show that the $2$-for-$3$  property of $\E$ implies that $X\to 0$ belongs to $\E$. Consider now the following homotopy cartesian square:
\[
\xymatrix{
X\ar@{}[rd]|\square\ar[r]\ar[d]&0\ar[d]\\
0\ar[r]&\Sigma X
}
\]
By Proposition \refbf{closure_phfs}, the map $0\to \Sigma X$ belongs to $\E$, that is $\Sigma(0/\E)\subseteq 0/\E$. One verifies similarly that $\M/0\subseteq \Sigma(\M/0)$.
\item Let $X\in \cD$, consider a factorization of the map $X\to 0$ as follows:
\[
X \xto{e} T \xto{m}  0\ \ \text{ with }\ \ e\in \E,\ m\in \M.
\] 
Now we can complete the map $e$ to a triangle to get
\[
R\to X \xto{e} T\to \Sigma R.
\]
By the normality of $\F$, $R\in 0/\E$ and $T\in \M/0$. \qedhere
\end{itemize}
\end{proof}




\begin{theorem}[the triangulated Rosetta stone]\label{triang-rosetta}
Let $\cD$ be a triangulated category, then there is a bijective correspondence
\[
\xymatrix@R=0pt{
\Phi:\left\{
{\begin{smallmatrix}
\text{normal triangulated}\\
\text{\textsc{tth}s on }\cD
\end{smallmatrix}}
\right\}
\ar@{<->}[rr]&&
\left\{
{\begin{smallmatrix}
\text{$t$-structures}\\
\text{ on }\cD
\end{smallmatrix}}
\right\}:\Psi\\
(\E,\M) \ar@{|->}[rr]&& \Big(0/\E, \Sigma(\M/0)\Big) \\
(\E_\t,\M_\t) \ar@{<-|}[rr]&& \t.
}
\]
%
%
%\[
%\begin{array}{ccc}
%\Phi\colon\textsc{n}\text{\htth}(\cD) & \leftrightarrows & \textsc{ts}(\cD)\colon \Psi\\
%(\E,\M) & \overset{}\mapsto & 0/\E \\
%(\E_\t,\M_\t) & \overset{}\mapsfrom & \t
%\end{array}
%\]
%Then, $\Phi$ and $\Psi$ are inverse bijections.
\end{theorem}
\begin{proof}
We have already verified in the previous subsections that $\Phi$ and $\Psi$ are well-defined. Consider now a $t$-structure $\t$ and let us show that $\t=\Phi\Psi\t$, that is, we should verify that $ \cD^{\leq 0} =0/\E_\t$. But this is true since clearly $X\in  \cD^{\leq 0} $ if and only if $0\to X$ belongs to $\E_\t$, that is, $X\in 0/\E_\t$. 

On the other hand, let $\F=(\E,\M)$ and let us show that $\F=\Psi\Phi\F$. Let $\phi\in \E_{\t_\F}$, that is, $\phi^{\geq1}$ is an isomorphism and consider the following commutative square:
\[
\xymatrix{
X\ar[r]^{\rho_X}\ar[d]_\phi&X^{\geq1}\ar[d]^{\phi^{\geq1}}\\
Y\ar[r]^{\rho_Y}&Y^{\geq1}
}
\]
Notice that $\rho_X$ and $\rho_Y$ belong to $\E$ (in fact these reflections are constructed taking an $\F$-factorization of the final maps $X\to 0$ and $Y\to 0$, see the last part of the proof of Lemma \refbf{htt_induces_t_structure}). The composition $\rho_Y\phi=\phi^{\geq1}\rho_X$ belongs to $\E$ since $\phi^{\geq1}\in \E$ (as $\E$ contains any isomorphism) and we have already observed that $\rho_X\in \E$. For the $3$-for-$2$ property this means that $\phi\in \E$. This shows that $\E_{\t_\F}\subseteq \E$. One proves in the exact same way that $\M_{\t_\F}\subseteq \M$, but these two conditions together mean that $\F=\F_{\t_\F}$, as desired.\end{proof}
