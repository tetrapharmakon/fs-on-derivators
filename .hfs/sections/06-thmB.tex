\section{Coherence of factorization algebras}
\label{sec:thmB}

This last section of the paper is devoted to introduce all the background needed to discuss, precisely state and, finally, prove \athm \textbf{IV} of the Introduction. First of all, in §\refbf{monads_subs}, we recall from \cite[§\textbf{1}]{lack2002codescent} the relevant definitions of 2-monads and pseudo-algebras. After that, in §\refbf{squaring_subs}, we specialize these general definitions to the so-called squaring monad on $\PDer$. \athm\@\textbf{IV} is then proved at the end of §\refbf{higher_coherence_sub}.
 
\subsection{2-monads}\label{monads_subs}
One of the most annoying features of higher dimensional monad theory is in how many place the coherence conditions can hide: the category $\cate{K}$ where the monad is defined, the monad $T$ itself, the naturality for multiplication and unit, and their associativity and unitality constraints, as well as the compatibility conditions for a $T$\hyp{}algebra, can all give rise to some diagrams that commute only up to a (invertible or non\hyp{}invertible) 2-cell.

Of course, some of these combinations of laxity are quite uncommon: 2-dimensional monad theory often copes with \emph{strict} 2-monads, or with strong monads that can be suitably ``strictified''. According to the existing zoology, here we need \emph{lax algebras for a strict pseudo-monad on a strict 2-category}. However, having no interest in different flavours, we simply call it the category of ``algebras for a 2-monad $T$''. We start with the definition of 2-monad, from \cite{lack2002codescent}.
\begin{definition}[2-monad]\label{two-monad}
Let $\cate{K}$ be a strict 2-category. A \emph{2-monad} on $\cate{K}$ consists of a tuple $\TT=(T,\mu,\eta,\ass, \uni)$ where $T$ is a strict endofunctor $T\colon\cate{K}\to\cate{K}$ endowed with a pair $(\mu,\eta)$ of 2-cells $\mu\colon T\circ T \Rightarrow T$, $\eta \colon \id_{\cate{K}}\Rightarrow T$ subject to the following relations:
\begin{itemize}
\item[(\textsc{mn})] the components of $\mu$ and $\eta$ fit into pseudo-commutative diagrams
\[
\vcenter{\xymatrix{
T^2K\ar[r]^{\mu_K}\ar@{}[dr]|{\Swarrow\me_f}\ar[d]_{T^2f} & TK\ar[d]^{Tf} & K\ar[r]^{\eta_K}\ar@{}[dr]|{\Swarrow\yu_f}\ar[d]_f & TK\ar[d]^{Tf}\\
T^2K' \ar[r]_{\mu_{K'}} & TK' & K' \ar[r]_{\eta_{K'}}& TK'
}}
\]
for 2-cells $\yu_f$ and $\me_f$ subject to the obvious conditions with respect to composition and identity 1-cells
 (these are nothing more than pseudo\hyp{}naturality conditions that can be be applied to any 2-cell).
\item[(\textsc{ma})] $\mu$ is associative, in that the diagram
\[
\xymatrix@R=1.5cm@C=1.5cm{
T^3K \ar[r]^{\mu_{TK}}\ar@{}[dr]|{\Swarrow\ass_K}\ar[d]_{T\mu_K} & T^2K \ar[d]^{\mu_K} \\
T^2K \ar[r]_{\mu_K}& TK
}
\]
commutes when filled by an invertible 2-cell $\ass_K \colon \mu_K\circ (\mu * T)_K \Rightarrow \mu_K \circ (T * \mu)_K$, which can be regarded as the $K$-component of an invertible 3-cell $\ass \colon \mu\circ (\mu *T) \Rrightarrow \mu\circ (T *\mu)$.
\item[(\textsc{mu})] $\eta$ is unital, in that the diagram
\[
\xymatrix@R=1.5cm@C=1.5cm{
TK \ar[r]^{\eta_{TK}}\ar@{-}[dr] \ar[d]_{T\eta_K}& T^2K\ar[d]^{\mu_K} \\
T^2K \ar[r]_{\mu_K} \ar@{}[ur]|(.35){\Swarrow\uni_{\textsc{l},K}}\ar@{}[ur]|(.65){\uni_{\textsc{r},K}\Nearrow}& TK
}
\]
commutes when filled with invertible 2-cells $\uni_{\textsc{r},K}\colon \id_{TK}\Rightarrow \mu_K\circ (\eta *T )_K$ and $\uni_{\textsc{l},K}\colon \id_{TK}\Rightarrow \mu_K\circ ( T * \eta )_K$, which can be regarded as the $K$-components of invertible 3-cells $\uni_\textsc{l} \colon \id_{T}\Rrightarrow \mu\circ ( T * \eta )$ and $\uni_\textsc{r} \colon \id_{T}\Rrightarrow \mu\circ ( \eta *T )$.
\end{itemize}
\end{definition} 
\begin{definition}[pseudo-algebras for a 2-monad]\label{two-algebras}
Let $\TT=(T,\mu,\eta,\ass,\uni)$ be a 2-monad on $\cate{K}$. A \emph{2-algebra} for $\TT$, or a $\TT$-algebra for short, consists of a tuple $\underline{A}=(a, \alpha_m, \alpha_u)$ where $a\colon TA\to A$ is a 1-cell of $\cate{K}$, and $\alpha_m,\alpha_u$ are invertible 2-cells called respectively the \emph{extended associator} and the \emph{normalizer} of the algebra structure, such that the following diagrams of 2-cells commute:
\begin{gather}
\vcenter{\xymatrix@C=.4cm{
&T^2A \ar[rr]^{Ta} \ar@{}[dd]|{\Downarrow\me_a}\ar[dr]_{\mu_A}&& TA\ar[dr]^a \ar@{}[dl]|{\Swarrow\alpha_m}\\
T^3A \ar[ur]^{T^2a}\ar[dr]_{\mu_{TA}}&& TA \ar@{}[dr]|{\Searrow\alpha_m} \ar[rr]_a&& A\\
& T^2A \ar[ur]^{Ta}\ar[rr]_{\mu_A}&& TA\ar[ur]_a
}}
\quad
{\Huge =}
\quad 
\vcenter{\xymatrix@C=.3cm{
&T^2A \ar@{}[dr]|{T\alpha_m\Searrow}\ar[rr]^{Ta}&& TA\ar@{}[dd]|{\Downarrow\alpha_m}\ar[dr]^{a} \\
T^3A \ar[rr]^{T\mu_A}\ar[ur]^{T^2a}\ar[dr]_{\mu_{TA}}&& T^2A\ar@{}[dl]|{\ass_A\Swarrow}\ar[ur]^{Ta}\ar[dr]_{\mu_A} && A\\
& T^2A \ar[rr]_{\mu_A}&& TA\ar[ur]_a
}}\notag\\[5mm]
\vcenter{\xymatrix@C=.4cm{
&A \ar@{}[dd]|{\Downarrow\yu_m}\ar@{-}@/^1.5pc/[drrr]\ar[dr]_{\eta_A}&&\\
TA \ar[ur]^a\ar[dr]_{\eta_{TA}}&& TA\ar@{}[ur]|{\Swarrow\alpha_u}\ar@{}[dr]|{\Searrow\alpha_m} \ar[rr]_a&& A\\
& T^2A \ar[ur]^{Ta}\ar[rr]_{\mu_A}&& TA\ar[ur]_a
}}
\quad
{\Huge =}
\quad 
\vcenter{\xymatrix@C=.4cm{
&A \ar@{}[rrrdd]|{=}\ar@{-}@/^1.5pc/[drrr]&&\\
TA  \ar@{-}@/^1.5pc/[drrr] \ar[dr]_{\eta_{TA}}\ar[ur]^a&& && A\\
&T^2A \ar[rr]_{\mu_A} \ar@{}[ur]|{\Swarrow\uni_{\textsc{r},A}}&& TA\ar[ur]_a &
}}\notag\\[5mm]
\vcenter{\xymatrix@C=.3cm{
&& & TA\ar[dr]^a\ar@{}[dd]|{\Downarrow\alpha_m}\\
TA\ar[rr]^{T\eta_A} \ar@{-}@/^1.5pc/[urrr]\ar@{-}@/_1.5pc/[drrr]&& T^2A \ar[ur]^{Ta}\ar[dr]_{\mu_A}
\ar@{}[ul]|{\Searrow T\alpha_u}\ar@{}[dl]|{\Nearrow \uni_{\textsc{l},A}}&& A\\
&& & TA\ar[ur]_a
}}
\quad
{\Huge =}
\quad 
\vcenter{\xymatrix@C=.4cm{
&& & TA\ar[dr]^a\\
TA \ar@{}[rrrr]|{||} \ar@/^1.5pc/[urrr]\ar@/_1.5pc/[drrr] && && A\\
&& & TA\ar[ur]_a
}}
\end{gather}
\end{definition}
\begin{remark}
We often stick to denote a pseudo-algebra for a monad $\TT$ simply as a \emph{$T$-algebra}; we also call \emph{normal} a $T$-algebra for which the normalizer $\alpha_u$ is the identity map (so $a\circ \eta_A = \id_A$ and the coherence diagrams above obviously simplify). We will be mainly interested in normal $T$-algebras; this is not restrictive, as shown in \refbf{normal_not_restrictive}.
\end{remark}
%We can then define
%\begin{definition}[lax morphism of $T$-algebras]\label{two-algebra-morphism}
%Let $\underline{A} = (a,\alpha_m,\alpha_u)$ and $\underline{B} = (b, \beta_m,\beta_u)$ be two $T$-algebras for a 2-monad $(T,\mu,\eta,\ass,\uni)$. A morphism of $T$-algebras consists of a pair $(f,\fe)$ where $f\colon A\to B$ is a 1-cell in $\cate{K}$ and $\fe\colon b\circ Tf \Rightarrow f\circ a$ is a 2-cell such that the following two diagrams of 2-cells commute:
%\begin{gather}
%\vcenter{\xymatrix@C=.35cm{
%& T^2B \ar[rr]^{Tb}\ar[dr]_{\mu_B}&& TB\ar[dr]^b\ar@{}[dl]|{\Swarrow\beta_m} \\
%T^2A \ar@{}[rr]|{\Downarrow\me_f}\ar[dr]_{\mu_A}\ar[ur]^{T^2f}&& TB \ar[rr]^b \ar@{}[dr]|{\Searrow\fe} && B\\
%& TA\ar[ur]^{Tf} \ar[rr]_a && A\ar[ur]_f
%}}
%\quad
%{\Huge =}
%\quad 
%\vcenter{\xymatrix@C=.35cm{
%& T^2B \ar[rr]^{Tb} \ar@{}[dr]|{\Searrow T\fe} && TB\ar[dr]^b \\
%T^2A\ar[dr]_{\mu_A}\ar[ur]^{T^2f} \ar[rr]^{Ta} && TA\ar@{}[rr]|{\Downarrow\fe}\ar@{}[dl]|{\Downarrow\alpha_m} \ar[ur]^{Tf}\ar[dr]_a && B\\
%& TA  \ar[rr]_a&& A\ar[ur]_f
%}}
%\notag\\
%\vcenter{\xymatrix@C=.5cm{
%& B \ar@{-}@/^1.5pc/[drrr]\ar[dr]^{\eta_B} \ar@{}[dd]|{\Downarrow\me_f} &&\\
%A \ar[dr]_{\eta_A}\ar[ur]^f&& TB \ar[rr]_b\ar@{}[ur]|{\Swarrow\beta_u} \ar@{}[dr]|{\Searrow\fe} && B\\
%&TA\ar[ur]^{Tf} \ar[rr]_a&& A\ar[ur]_f
%}}
%\quad
%{\Huge =}
%\quad 
%\vcenter{\xymatrix@C=.5cm{
%& B \ar@{-}@/^1.5pc/[drrr]\ar@{}[ddrrr]|{=}\\
%A \ar@{-}@/^1.5pc/[drrr]\ar[dr]_{\eta_A}\ar[ur]^f&& && B\\
%&TA \ar@{}[ur]|{\Swarrow\alpha_u} \ar[rr]_a&& A\ar[ur]_f & 
%}}
%\label{alg-constr}
%\end{gather}
%\end{definition}
%\begin{remark}
%Notice that algebra morphisms are \emph{lax}, in that the 2-cell $\fe$ is not invertible. We need this weakness as we will ultimately characterize lax $(\firstblank)^\due$-algebras as functors that preserve the right class $\M$ of a factorization system $(\E,\M)$, and \emph{colax} $(\firstblank)^\due$-algebras as functors that preserve the left class. This follows from a simple argument: for any adjunction $F \dashv U \colon K \leftrightarrows K'$ between objects that are algebras for a 2-monad, $F$ defines a colax morphism of $T$-algebras if and only if $U$ defines a lax morphism of $T$-algebras.
%\end{remark}
%
\subsection{The squaring monad and its algebras}\label{squaring_subs}
%With these notations it is now easy to introduce the squaring monad as follows:
\begin{definition}[the squaring monad on $\PDer$]\label{def_squaring_monad}
Let $\Dia$ be a $2$-category of diagrams and denote by $\PDer$ the 2-category of pre-derivators of type $\Dia$. The \emph{squaring monad} on $\PDer$ is the triple $((-)^\due, \Delta^\circledast,\pt^\circledast)$, where $(-)^\due \coloneqq \textsf{sh}(\due,-)$, while $\Delta$ and $\pt$ are defined in Section \refbf{comonoid_due_subs}. 
\end{definition}
Using the shift functor $\textsf{sh}(-,-)$ to transport the comonoid structure on $\due$ described in §\refbf{comonoid_due_subs}, one can see that the squaring monad is a 2-monad in the sense of \cite{lack2002codescent} but in a very strict sense, in that we have equalities (and not mere natural isomorphisms) in the following expressions
\[
\begin{cases}
\Delta^\circledast\circ(\pt\times \id_{\due})^\circledast=\id_{\D^\due}=\Delta^\circledast\circ(\id_{\due}\times \pt)^\circledast\notag\\
\Delta^\circledast\circ(\Delta\times\id_{\due})^\circledast=\Delta^\circledast\circ(\id_{\due}\times \Delta)^\circledast.
\end{cases}
\]
The strictness of the squaring monad can be used to greatly simplify the definition of pseudo-algebras given in \cite{lack2002codescent}:
\begin{definition}\label{def_factorization_alg}
In the same setting of \adef\refbf{def_squaring_monad}, let $\D$ be a pre-derivator of type $\Dia$. A \emph{normal pseudo-algebra} for the squaring monad is a morphism $F\colon \D^\due\to \D$ such that $F\circ \pt^\circledast=\id_\D$, together with a natural isomorphism $\gamma\colon FF^\due\xto{\cong} F\Delta^\circledast$ that satisfies the following properties:
\begin{enumerate}
\item $\gamma *(\id_\due\times\pt)^\circledast=\id_F$;
\item $\gamma*(\pt\times \id_\due)^\circledast=\id_F$;
\item  $(\gamma*(\Delta\times\id_\due)^\circledast) \circ (\gamma* F^{\due\times \due})=(\gamma*(\id_\due\times\Delta)^\circledast) \circ  (F*\gamma^\due)$.
\end{enumerate}
We will refer to a normal pseudo-algebra over the squaring monad simply as a \emph{normal $(\firstblank)^\due$\hyp{}algebra}.
\end{definition}
\subsection{Coherence for factorization algebras}\label{higher_coherence_sub}
First of all, we are going to re-enact some technical results of \cite{RW}, in preparation for the proof of \athm\textbf{IV}; these are simply the result of having adapted the most relevant results in \cite[§\textbf{2}]{RW} from the 2-category $\Cat$ to the 2-category $\PDer$. 
%The first result we need holds in any 2-category, as a consequence of the interchange law between horizontal and vertical composition.


\begin{lemma}\label{a_straightforward_lemma}\label{nuff_to_determine}
Let $F\colon \D^\due\to\D$ be a normal factorization pre\hyp{}algebra; then precomposition with $l^{\circledast}$ induces a bijection
\[
\PDer(\D^{\due\times\due}, \D)(FF^\due, F\Delta^\circledast) \xto{\quad \firstblank * l^{\circledast}\quad} \PDer(\D^\due, \D)(FF^\due l^{\circledast}, F).
\]
\end{lemma}
\begin{proof}
We start noticing that, in any 2-category, given a diagram of 2-cells
\[
\xymatrix@C=2cm{
	A \ar@{}[r]|{\Downarrow\sigma}\ar@/^1pc/[r]^F\ar@/_1pc/[r]_G& B \ar@{}[r]|{\Downarrow\tau}\ar@/^1pc/[r]^S\ar@/_1pc/[r]_T & C
}
\]
if $T * \sigma$ is invertible, then $\tau  *F$ is determined by $\tau *G$, in the sense that $\tau * F = (T * \sigma)^{-1}\circ (\tau *G)\circ (S* \sigma)$. Similarly, if $S * \sigma$ is invertible then $\tau *G = (T *\sigma)\circ (\tau *F )\circ (S * \sigma)^{-1}$ (specialized to the 2-category $\Cat$, this is \cite[Lemma 2.2]{RW}).

For any $\tau\colon FF^\due\to F\Delta^\circledast$ there is a diagram of 2-cells in $\PDer$:
\[
\xymatrix@C=2cm{
	\D^{\due\times\due} \ar@{}[r]|{\Downarrow\eta_l^\circledast}\ar@/^1pc/[r]^{\id_{\D^{\due\times \due}}}\ar@/_1pc/[r]_{l^\circledast\Delta^{\circledast}}& \D^{\due\times\due} \ar@{}[r]|{\Downarrow\tau}\ar@/^1pc/[r]^{FF^\due}\ar@/_1pc/[r]_{F\Delta^{\circledast}} & \D
}
\]
Using the above general fact, one can easily prove that there is a bijection 
\[
\PDer(\D^{\due\times\due}, \D)(FF^\due, F\Delta^\circledast) \xto{\quad \firstblank * l^{\circledast}\Delta^{\circledast}\quad} \PDer(\D^{\due\times \due}, \D)(FF^\due l^{\circledast}\Delta^{\circledast}, F\Delta^{\circledast}),
\]
which induces the desired bijection since $\Delta^{\circledast}$ is co-fully faithful (for more details see \cite[§\textbf{2}]{RW}).
\end{proof}




\begin{lemma}
\label{iso_implies_EM}
Let $F\colon \D^\due\to \D$ be  a normal factorization pre\hyp{}algebra and suppose that there is an isomorphism $\alpha\colon FF^\due\to F\Delta^\circledast$. Then all the 2-cells $m_{e_{\firstblank}}$, $e_{m_{\firstblank}}$, $FF^\due*(\mu')^\circledast$ and $FF^\due*(\nu')^\circledast$ are invertible. As a consequence, $\F \coloneqq (\mathbb E_F,\mathbb M_F)$ is an Eilenberg\hyp{}Moore factorization system.
\end{lemma}
\begin{proof}
Consider the following commutative diagram, obtained applying $\alpha$ to the first diagram in \refbf{diamond_def}
\begin{equation}
\label{two_diamonds}
\xymatrix{
FF_rX\ar[dr]|{\alpha * r^\circledast}\ar[rr]^{m_{e_X}}\ar[dd]_{FF^\due * (\mu')^\circledast} && FX\ar[dd]^{e_{m_X}} \ar[dl]|{\alpha * v^\circledast}\\
& FX & \\
FX \ar[rr]_{FF^\due *(\nu')^\circledast}\ar[ur]|{\alpha * h^\circledast}&& FF_l X\ar[ul]|{\alpha * l^\circledast}
}
\end{equation}
where the central object results as a square of identities $\id_{FX}$ obtained from \refbf{properties_of_functors_among_ord}.($iv$). It is then clear that, being $\alpha$ invertible, the four arrows that point to $FX$ are all invertible, and so are the components of $m_{e_{\firstblank}}$, $e_{m_{\firstblank}}$, and $FF^\due*(\mu')^\circledast, FF^\due*(\nu')^\circledast$. 

For the last statement we should verify that $F_rX\in \mathbb E_F$ and $F_lX\in \mathbb M_F$, for any $X\in \D^\due(I)$, equivalently, one should verify that $F_lF_rX$ and $F_rF_lX$ are isomorphisms. But this is clear since the underlying diagram of $F_lF_rX$ is exactly $m_{e_X}$, while the underlying diagram of $F_rF_lX$ is $e_{m_X}$. 
\end{proof}





The above two lemmas were general facts about $\PDer$. From now on, we will need to work under much stronger hypotheses, indeed, we will need to assume that our pre-derivator $\D$ is either representable or that it is a stable derivator. In fact, the unique point in which we will actively use these hypotheses is in the following lemma, which is the counterpart of \cite[\acor\textbf{2.9}]{RW}.




\begin{lemma}
\label{tfae_for_gamma}
Suppose $\D$ is either a representable pre-derivator or a stable derivator. 
Let $F\colon \D^\due\to \D$ be  a normal factorization pre\hyp{}algebra and suppose that there is an isomorphism $\alpha\colon FF^\due\to F\Delta^\circledast$. Then,  $m_{e_{\firstblank}} =FF^\due*(\mu')^\circledast$ and $e_{m_{\firstblank}} =FF^\due*(\nu')^\circledast$. Furthermore, the following conditions are equivalent
\begin{enumerate}[label=($\roman*$)]
\item $\alpha*v^\circledast=\id_{F}$;
\item $\alpha*l^\circledast=(e_{m_{\firstblank}} )^{-1}$;
\item $\alpha*h^\circledast=\id_{F}$;
\item $\alpha*r^\circledast=m_{e_{\firstblank}}$.
\end{enumerate}
\end{lemma}
\begin{proof}
By Lemma \refbf{iso_implies_EM}, $F$ is an \textsc{em} factorization and so, by our hypotheses on $\D$ and the results in §\refbf{sec:squaring}, $\Psi_\F$ is fully faithful. Now consider the following functors (compare with \eqref{diamond_def})
\[
\begin{matrix}\xymatrix{
&V\ar@{=>}[dr]^{\underline\nu}\\
R\ar@{=>}[dr]_{\underline\mu'}\ar@{=>}[ur]^{\underline\mu}&&L\\
&H\ar@{=>}[ur]_{\underline\nu'}
}\end{matrix}
\colon \tre\times \due\times \due\longrightarrow  \tre\times\due
\]
where
\[
R(a,b,c) \coloneqq \begin{cases}
(0,0)&\text{if $a=0$;}\\
(1,r(b,c))&\text{if $a=1$;}\\
(2,b)&\text{if $a=2$;}
\end{cases}
\qquad
 L(a,b,c) \coloneqq \begin{cases}
(0,b)&\text{if $a=0$;}\\
(1,l(b,c))&\text{if $a=1$;}\\
(2,1)&\text{if $a=2$;}
\end{cases}
\]
\[
V(a,b,c) \coloneqq \begin{cases}
(0,0)&\text{if $a=0$;}\\
(1,v(b,c))&\text{if $a=1$;}\\
(2,1)&\text{if $a=2$;}
\end{cases}
\qquad
 H(a,b,c) \coloneqq \begin{cases}
(0,0)&\text{if $a=0$;}\\
(1,h(b,c))&\text{if $a=1$;}\\
(2,1)&\text{if $a=2$;}
\end{cases}
\]
and where $\underline\mu$, $\underline\mu'$, $\underline\nu$ and $\underline\nu'$ are defined in the unique possible way. Also notice that $V$ and $H$ are constructed in such a way that $V\circ ((0,2)\times \id_{\due\times \due})=H\circ ((0,2)\times \id_{\due\times \due})$ and
\[
\underline\mu*((0,2)\times \id_{\due\times \due})=\underline\mu'*((0,2)\times \id_{\due\times \due}).
\]
Given $X\in \D^\due(I)$, let $Y \coloneqq (0\times\id_\due)_!X\in \D^{\tre\times \due}(I)$ and notice that
\begin{align*}
\Psi^\due_\F F^{\tre\times\due}*\underline \mu^\circledast_Y&=((0,2)\times\id_\due)^\circledast F^{\tre\times\due}*\underline \mu^\circledast_Y\\
&=F^{\due\times\due}((0,2)^\due\times\id_{\due\times\due})^\circledast*\underline \mu^\circledast_Y\\
&=F^{\due\times\due}((0,2)^\due\times\id_{\due\times\due})^\circledast*(\underline \mu')^\circledast_Y\\
&=((0,2)\times\id_\due)^\circledast F^{\tre\times\due}*(\underline \mu')^\circledast_Y=\Psi^\due_\F F^{\tre\times\due}*(\underline \mu')^\circledast_Y.
\end{align*}
Since $\Psi_\F$ is faithful, we get that $F^{\tre\times\due}*\underline \mu^\circledast_Y=F^{\tre\times\due}*(\underline \mu')^\circledast_Y$. So in particular,
\begin{align*}
F^\due*\mu^\circledast_X&=F^{\due}(1\times \id_{\due\times\due})^\circledast *\underline \mu^\circledast_Y=(1\times \id_\due)^\circledast F^{\tre\times\due}*\underline \mu^\circledast_Y\\
&=F^{\tre\times\due}*(\underline \mu')^\circledast_Y=(1\times \id_\due)^\circledast F^{\tre\times\due}*(\underline \mu')^\circledast_Y\\
&=F^{\due}(1\times \id_{\due\times\due})^\circledast *(\underline \mu')^\circledast_Y=F^\due*(\mu')^\circledast_X.
\end{align*}
A similar argument shows that $F^\due*\nu^\circledast =F^\due*(\nu')^\circledast$. With these equalities, it is not difficult to derive the equivalence among (i), (ii), (iii) and (iv) by just looking at the commutative diagram in the proof of Lemma \refbf{iso_implies_EM}.
\end{proof}


At this point we can finally prove our \athm\textbf{IV}. The arguments used in the proof are analogous to those of  \cite[\athm\textbf{2.10} and \athm\textbf{2.11}]{RW}.

\begin{proof}[Proof of \athm\textbf{IV}]
The implication (1)$\Rightarrow$(2) is trivial, while (2)$\Rightarrow$(3) follows by Lemma \refbf{iso_implies_EM}. Hence, we assume (3)  and we show how to construct an extended associator 
\[
\alpha\colon FF^\due \xto{\sim} F\Delta^\circledast
\] 
that satisfies conditions \refbf{def_factorization_alg}.(1--3). Consider the following commutative diagrams
\[
\vcenter{
\xymatrix{
& \Delta r \ar[dr]^{\Delta* \mu}\ar[dl]_{\epsilon_r}& \\
\id_{\due\times \due}\ar[dr]_{ \eta_l}&& \Delta h\ar[dl]^{ \Delta*\nu}\\
& \Delta l}
%\xymatrix{
%0^\circledast \Delta^\circledast \ar[r]\ar[d]& F(0\times \id_\due)\ar[r]\ar[d] & F\Delta^\circledast \ar[d]^{FF^\due * \nu^\circledast}\\
%FF_r \ar[r]^{FF^\due * \epsilon_R}\ar[d]_{FF^\due * \mu^\circledast}& FF^\due \ar[r]^{FF^\due * \eta_L}\ar[d]& FF_l \ar[d]\\
%F\Delta^\circledast \ar[r]_{(\star\star)}& F(1\times\id_\due)\ar[r] & 1^\circledast\Delta^\circledast
%}
}
\qquad\qquad\vcenter
{\xymatrix{
& FF^\due (\Delta r)^\circledast \ar[dr]^{FF^\due * \mu^\circledast*\Delta^{\circledast}}\ar[dl]_{FF^\due * \epsilon_r^{\circledast}}& \\
FF^\due \ar[dr]_{FF^\due * \eta_l^\circledast}&& F\Delta^\circledast\ar[dl]^{FF^\due * \nu^\circledast*\Delta^{\circledast}}\\
& FF^\due (\Delta l)^\circledast
}}
\]
where the second one is obtained from the first one composing with $FF^\due$, after having applied $\D$. The \textsc{em} condition tells us the $FF^\due*\mu^\circledast$ and $FF^\due*\nu^\circledast$ are invertible. We claim that also the two remaining arrows in the above diagram are invertible. Let us give an argument just for $FF^\due * \epsilon_r^{\circledast}$ as $FF^\due * \eta_l^{\circledast}$ is an iso for formally dual reasons. By \refbf{ortho_to_self}, it is enough to show that $FF^\due * \epsilon_r^{\circledast}\in \E_F\cap \M_F$. Consider the following commutative diagrams:
\[
\xymatrix{
(0,0)\circ\pt\circ(\id_\due\times \pt)\ar[r]\ar[d]&((0,0),(0,1))\circ(\id_\due\times \pt)\ar[d]\\
\Delta\circ r\ar[r]&\id_{\due\times \due}}
\]
\[
\xymatrix{
(0,0)^\circledast\ar[r]^(.39){\star}\ar[d]_{\star}&F ((0,0),(0,1))^\circledast\ar[d]^{\star}\\
FF_r\Delta^\circledast\ar[r]&FF^2}
\]
where $(0,0)\colon \uno\to \due\times \due$ and $((0,0),(0,1))\colon \due\to\due\times \due$ select respectively the upper left corner and the left horizontal arrow in $\due\times \due$; furthermore, the arrows in the first diagram are the unique possible, while the second diagram is obtained from the first one composing with $FF^\due$, after having applied $\D$. Using the \textsc{em} condition, one can show that the arrows marked by ($\star$) are in $\E_F$ (since they are instances of the natural transformation $e_{-}$) and so, by the closure and cancellation properties of this class, also the remaining arrow, which is $FF^\due * \epsilon_r^{\circledast}$, does belong to $\E_F$. A similar argument but starting with the following diagram 
\[
\xymatrix{
\Delta\circ r\ar[r]\ar[d]&\id_{\due\times \due}\ar[d]\\
\Delta\circ (\id_\due\times \pt)\ar[r]&((1,0),(1,1))\circ(\id_\due\times \pt)}
\]
shows that $FF^\due * \epsilon_r^{\circledast}$ does belong to $\M_F$. We can now define 
\[
\alpha  \coloneqq  (FF^\due * \mu^\circledast*\Delta^\circledast) \circ (FF^\due * \epsilon^\circledast_r)^{-1}
\] 
and verify that it is the extended associator we are looking for. 
In fact, it is  easy to show that $\alpha$ satisfies one of the equivalent conditions of \refbf{tfae_for_gamma}, so that it satisfies conditions (1) and (2) of \refbf{def_factorization_alg}, as they are exactly (i) and (iii) of \refbf{tfae_for_gamma}.  
%
%It is easy to see that the diagram of 2-cells
%\[
%\xymatrix@C=2cm{
%	FF^\due R\Delta^\circledast \ar[r]^{FF^\due*\epsilon_R}\ar[d]_{\alpha*R\Delta^\circledast}& FF^\due \ar[r]^{FF^\due*\eta_L}\ar[d]^\alpha & FF^\due L\Delta^\circledast \ar[d]^{\alpha*L\Delta^\circledast}\\
%	F\Delta^\circledast\ar@{=}[r] & F\Delta^\circledast \ar@{=}[r] & F\Delta^\circledast 
%}
%\]
%commutes (it is the perfect analogue in $\PDer$ of the diagram before \cite[Lemma \textbf{2.6}]{RW}), we see that $\alpha = (\alpha*L\Delta^\circledast) \circ (FF^\due*\eta_L)$. This is also equal to $\dots$, using diagram (\refbf{two_diamonds}). As a consequence, each component of $\alpha$ is determined by the composition $(\alpha * L\Delta^\circledast)\circ (FF^\due * \eta_L)$, and if $\alpha$ is invertible, then so are the 2-cells $FF^\due*\epsilon_R$ and $FF^\due*\eta_L$.
%
It remains to check condition \refbf{def_factorization_alg}.(3). %, namely the fact that $F$ is endowed with an extended associator and that its coherence comes for free. 
For this, consider the following diagram of 2-cells
\begin{equation}\label{diff_coh_dia}
\xymatrix@C=2cm{
\ar[d]_{F*\alpha^\due} \ar[r]^{\alpha * F^{\due\times\due}}FF^\due F^{\due\times\due} & **[r] F\Delta^\circledast_\D F^{\due\times\due} \overset{\star}= FF^\due \Delta^\circledast_{\D^\due} \ar[d]^{\alpha  * \Delta^\circledast_{\D^\due}}\\
FF^\due (\Delta^\circledast_\D)^\due \ar[r]_{\alpha * (\Delta^\circledast_\D)^\due} & **[r] F\Delta^\circledast_\D \Delta^\circledast_{\D^\due} \overset{\star\star}= F\Delta^\circledast_\D(\Delta^\circledast_\D)^\due
}
\end{equation}
(the equality ($\star$) follows from naturality, and the equality ($\star\star$) follows from associativity). Notice that \refbf{def_factorization_alg}.(3) expresses exactly the commutativity of the above square. Thanks to \refbf{nuff_to_determine} (applied twice), it is enough to check that it commutes when composed in the north\hyp{}west corner with $l_{\D^\due}^{\circledast}l^{\circledast}$. Notice that, after composing with $l_{\D^\due}^{\circledast}$, the paths going from the north\hyp{}west to the south\hyp{}east corner are both 2-cells going from $FF^\due F^{\due\times \due}$ to $F\Delta_\D^\circledast \Delta^\circledast_{\D^\due}$. Let us fix an arbitrary 2-cell 
\[
\theta\colon FF^\due F^{\due\times \due}\to F\Delta_\D^\circledast \Delta^\circledast_{\D^\due}
\]
and let us record some of its general properties. We first consider 
\[
\xymatrix@C=2.5cm{
	\D^\due 
	\ar@/^1pc/[r]^{v^\circledast}
	\ar@/_1pc/[r]_{l^\circledast}
	\ar@{}[r]|{\Downarrow\nu^\circledast}
	& \D^{\due\times\due} 
	\ar@/^1pc/[r]^{FF^\due F^{\due\times\due} l_{\D^\due}^{\circledast}}
	\ar@/_1pc/[r]_{F\Delta^\circledast_\D}
	\ar@{}[r]|{\Downarrow\theta *l^{\circledast}_{\D^\due}}
	& \D
}
\]
We claim that the whiskering $FF^\due F^{\due\times\due} l^\circledast_{\D^\due} * \nu^{\circledast}$ is invertible, in fact,
\begin{align*}
FF^\due F^{\due\times\due} l^\circledast_{\D^\due}  * \nu^\circledast &\cong FF^\due l^\circledast  F^\due * \nu^\circledast\\
&= FF_lF^\due * \nu^\circledast \\
&\cong  FF^\due * \nu^\circledast
\end{align*}
and we have already noticed that this last 2-cell is invertible. Hence,
\[
\theta *l^{\circledast}_{\D^\due}l^\circledast=(F\Delta^\circledast_\D*\nu^\circledast)\circ (\theta *l^{\circledast}_{\D^\due}v^{\circledast})\circ(FF^\due F^{\due\times\due} l_{\D^\due}^{\circledast}*\nu^\circledast)^{-1}
\]
This shows that, given $\theta,\, \theta'\colon FF^\due F^{\due\times \due}\to F\Delta_\D^\circledast \Delta^\circledast_{\D^\due}$, then $\theta *l^{\circledast}_{\D^\due}l^\circledast=\theta' *l^{\circledast}_{\D^\due}l^\circledast$ if and only if $\theta *l^{\circledast}_{\D^\due}v^{\circledast}=\theta' *l^{\circledast}_{\D^\due}v^{\circledast}$. This reduces the verification of the commutativity of \eqref{diff_coh_dia}, to the proof of the following equality:
\[
((\alpha * \Delta_{\D^\due}^\circledast)\circ (\alpha * F^{\due\times\due})) *l^{\circledast}_{\D^\due}v^{\circledast}=((\alpha * (\Delta_\D)^\due)\circ (F * \alpha^\due))*l^{\circledast}_{\D^\due}v^{\circledast}
\]
To conclude notice that, by \refbf{tfae_for_gamma}.(i), both sides are equal to $\id_{F}$.
\end{proof}


%
%\begin{theorem}
%\label{its_an_algebra_for_free}
%Let $F\colon \D^\due\to \D$ be a factorization pre\hyp{}algebra. If there exists an isomorphism $\alpha \colon FF^\due \to F\Delta^\circledast$ that satisfies any of the equivalent conditions of \refbf{tfae_for_gamma}, then the equations
%\begin{align*}
%\alpha &= (\alpha*l^\circledast\Delta^\circledast) \circ (FF^\due*\eta_l^\circledast)\\
%       &= (\alpha*r^\circledast\Delta^\circledast) \circ (FF^\due*\epsilon_r^\circledast)^{-1}
%\end{align*}
%hold, where $r^{\circledast}\adjunct{\eta_r^{\circledast}}{\epsilon_r^{\circledast}}\Delta^\circledast\adjunct{\eta_l^{\circledast}}{\epsilon_l^{\circledast}}l^{\circledast}$. Furthermore, $\alpha$ satisfies \emph{all} the axioms of \adef\refbf{def_factorization_alg}, so that $\alpha$ is an extended associator; this endows $F$ with a uniquely determined structure of normal $(\firstblank)^\due$-algebra.
%\end{theorem}
%\begin{proof}
%
%\end{proof}
%
%
%
%
%\begin{theorem}
%\label{has_alg_struct}
%Let $F\colon \D^\due\to \D$ be a normal factorization pre\hyp{}algebra. Then the following conditions are equivalent:
%\begin{itemize}
%	\item $F$ admits a (necessarily unique) $(\firstblank)^\due$\hyp{}algebra structure;
%	\item there exists an isomorphism $\alpha\colon FF^\due \xto{\cong} F\Delta^\circledast$ (that necessarily plays the r\^ole of the extended associator);
%	\item all the components of $m_{e_{\firstblank}}$ and $e_{m_{\firstblank}}$ are invertible.
%\end{itemize}
%\end{theorem}
%\begin{proof}
%Consider the diagrams
%\[
%\vcenter{\xymatrix{
%0^\circledast \Delta^\circledast \ar[r]\ar[d]& F(0\times \id_\due)\ar[r]\ar[d] & F\Delta^\circledast \ar[d]^{FF^\due * \nu^\circledast}\\
%FF_r \ar[r]^{FF^\due * \epsilon_R}\ar[d]_{FF^\due * \mu^\circledast}& FF^\due \ar[r]^{FF^\due * \eta_L}\ar[d]& FF_l \ar[d]\\
%F\Delta^\circledast \ar[r]_{(\star\star)}& F(1\times\id_\due)\ar[r] & 1^\circledast\Delta^\circledast
%}}\qquad\qquad
%\vcenter{\xymatrix{
%& FF_r \ar[dr]^{FF^\due * \mu^\circledast}\ar[dl]_{FF^\due * \epsilon_R}& \\
%FF^\due \ar[dr]_{FF^\due * \eta_L}&& F\Delta^\circledast\ar[dl]^{FF^\due * \nu^\circledast}\\
%& FF_l
%}}
%\]
%(the second is obtained joining together the north-east and south-west corners of the first, and it commutes by \cite[Lemma \textbf{2.1}]{RW}). The arrow $FF^\due * \epsilon_R$ is in $\E_F$ (by the closure properties of $\E_F$ and its definition as a \textsc{em} factorization system). Now the bottom line of the first diagram is in $\M_F$; but then so is the first arrow $(\star\star)$, by cancellation properties. This means that $FF^\due * \epsilon_R$ is in $\M_F\cap \E_F$, and then it is invertible by \refbf{ortho_to_self}.
%
%Now, define $\alpha = (FF^\due * \mu^\circledast) \circ (FF^\due * \epsilon_R)^{-1}$. This is an isomorphism $FF^\due \xto{\cong} F\Delta^\circledast$. It's easy to show that $\alpha$ satisfies one of the equivalent conditions of \refbf{tfae_for_gamma}, \eg that $\alpha * v^\circledast =  \id_F$; this entails that $(F,\alpha)$ is a $(\firstblank)^\due$\hyp{}algebra.
%\end{proof}
