\section{Functoriality of factorizations }\label{sec:thmA}
\setlength{\epigraphwidth}{.2\textwidth}
\epigraph{
	\begin{CJK}{UTF8}{min}道生一,\end{CJK}\\
	\begin{CJK}{UTF8}{min}一生二,\end{CJK}\\
	\begin{CJK}{UTF8}{min}二生三,\end{CJK}\\
	\begin{CJK}{UTF8}{min}三生万物。\end{CJK}
}{Laozi 42}
%
%In the following sections we extend the two main results of \cite{Korostenski199357} to the setting of coherent factorization systems on derivators. 
%
%As sketched in the introduction, the authors of \cite{Korostenski199357} consider a factorization functor for morphisms in a category $\C$, \ie a section of the composition functor $\C^\tre \to \C^\due$, that sends a morphism $f\colon X \to Y$ to an object $F(f)$ such that the composition $X\xto{e_f}F(f)\xto{m_f}Y$ is $f$.
%
%To such a functorial factorization one associates two classes of morphisms:
%\begin{gather}
%\E_F \coloneqq \{h\in \C^\due \mid m_h\text{ is an iso}\}\notag\\
%\M_F \coloneqq \{h\in \C^\due \mid e_h\text{ is an iso}\}.
%\end{gather}
%A functorial factorization as above is said to be an \emph{Eilenberg\hyp{}Moore factorization} provided $e_f\in \E_F$ and $m_f\in \M_F$ for any $f\in \C^\due$. Then, the authors are able to prove the following two theorems that justify this name:
%\begin{theorem*}\cite[\athm\textbf{A}]{Korostenski199357}
%Orthogonal factorization systems can be equivalently described as Eilenberg\hyp{}Moore factorizations.
%%weak factorization systems that satisfy the following property:
%%\begin{quote}
%%For each $f \colon X\to Y$ that $(\E,\M)$ factors as the composable pair $(e_f,m_f) \in \E\times \M$, the $\M$-part $m_{e_f}$ of $e_f$, as well as the $\E$-part $e_{m_f}$ of $m_f$, are isomorphisms.
%%\end{quote}
%\end{theorem*}
%
%We reformulate \athm\textbf{A} in the setting of derivators as follows:
%\begin{theorem}[theorem III]\label{algebras-in-disguise}
%Let $\D$ be a derivator. An Eilenberg\hyp{}Moore factorization $F\colon \D^\due\to \D$ (see \adef\refbf{def_EM_algebra}) induces two sub pre-derivators $\mathbb E_F$ and $\mathbb M_F\subseteq \D^\due$ such that $\mathbb E_F\corth\mathbb M_F$. If $\D$ is either discrete or stable, then $(\mathbb E_F,\mathbb M_F)$ is a \dfs (\adef\refbf{def_hfs}). 
%\end{theorem}
%We defer the proof of this statement to the end of this section.

Let $\Dia$ be a $2$-category of diagrams (see the beginning of §\refbf{sec:squaring}) and denote by $\PDer$ the 2-category of pre-derivators of type $\Dia$. Recall from \cite[\textbf{2.1}.(ii)]{Moritz} that there is a strict 2-functor, called \emph{shift functor}, defined as
\[
\textsf{sh}(-,-)\colon \Dia^\opp\times \Der\longrightarrow \Der\colon(J,\D)\mapsto \D^{J}.
\]
Here $\D^J$ is the pre-derivator such that $\D^J(I) \coloneqq \D(J\times I)$. Given a functor $u\colon I\to J$ in $\Dia$ the action of $\D^J$ is described by the formula $\D^K(u) \coloneqq \D(\id_K\times u)$ for all $K\in\Dia$, and similar formulas hold for the action on $\D^J$ on natural transformations. 

It is convenient for us to introduce the following notation: \cite{Moritz} blurs the distinction between the functor $u^*\colon \D(J)\to \D(I)$ image of $u\colon I\to J$ under a derivator $\D$ and the pseudo\hyp{}natural transformation $u^\circledast\colon \D^{J}\to \D^{I}$ induced by the same $u$. However, this clash of notation can be harmful to our discussion, since we will mainly consider instances of the second map, while needing a reference to its action on morphisms of $\Dia$.
\begin{notat}
Given a pre-derivator $\D$ and a functor $u\colon I\to J$ in $\Dia$, we let 
\[
u^\circledast  \coloneqq \textsf{sh}(u,\D)\colon \D^{J}\to \D^{I}.
\]
If $\D$ is a derivator, then $u^\circledast$ has a left and a right adjoint as a 1-cell in the 2-category $\PDer$, that we denote by
\[
u_\circledbang\dashv u^\circledast\dashv u_\circledast.
\]
Given $K\in \Dia$, the components $u_\circledbang(K),u_\circledast(K)\colon \D^{J}(K)\to \D^{I}(K)$ are given by
$u_\circledbang(K)=(u\times \id_K)_!$ and $u_\circledast(K)=(u\times \id_K)_*$.
\end{notat}
\subsection{Factorization pre-algebras}\label{subs_facprealg}
Following \cite{RW}, given a category $\C$, a functor $F\colon \C^\due\to \C$ such that $F(\id_c)=c$ for each $c\in \C$, and not only a coherent isomorphism $F(\id_c)\cong c$, is said to be a \emph{normal} factorization pre\hyp{}algebra. In this subsection we introduce a similar notion in the context of pre-derivators and we describe some of its elementary properties.
\begin{remark}
We explicitly remark that there is no connection between the normality of a factorization pre\hyp{}algebra and the normality of a (homotopy) torsion theory defined in \refbf{hontt}; the coincidence of the two terms is only an unfortunate clash of terminology of the two sources from which we are extracting our main theorems.
\end{remark}
\begin{definition}[normal factorization pre\hyp{}algebra]\label{def_factorization_prealgebra}
A morphism $F\colon \D^\due\to \D$ in $\PDer$
is said to be a \emph{factorization pre\hyp{}algebra} provided  there exists an isomorphism $\gamma\colon F\circ \pt^\circledast\to \id_{\D}$. A factorization pre\hyp{}algebra is \emph{normal} provided $F\circ \pt^\circledast=\id_{\D}$. 
\end{definition}
The reason to call these functors ``pre-algebras'' will be clarified in Section \refbf{sec:thmB}: factorization pre\hyp{}algebras are just algebras over the squaring monad deprived of their extended associator. On the other hand, the use of the term ``factorization'' is justified by the validity of the following lemma in the context of pre-derivators: we recall the adjunctions $1^\circledast \adjunct{\eta}{} \pt^\circledast  \adjunct{}{\epsilon}0^\circledast$ and define
\begin{lemma}
Let $\D$ be a pre-derivator and $F\colon \D^\due\to \D$ a normal factorization pre\hyp{}algebra. Then $F$ induces a factorization
\[
\xymatrix{
0^\circledast\ar[r]^{e_{\firstblank}}&F\ar[r]^{m_{\firstblank}}&1^\circledast
}
\]
 of the 2-cell $\kappa \colon 0^\circledast \to 1^\circledast\colon \D^\due \to \D$ introduced in \refbf{la-kappa}, where $e,m$ are obtained whiskering $F$ with the unit and counit above:
 \begin{align}
 e_{\firstblank} &\colon 0^\circledast=F \circ \pt^\circledast\circ  0^\circledast \xto{F * \epsilon} F\notag\\
 m_{\firstblank} &\colon F\xto{F * \eta} F \circ \pt^\circledast\circ 1^\circledast=1^\circledast.
 \end{align}
Conversely, for each pair of natural transformations $\ee \colon 0^\circledast \to F$ and $\mm\colon F \to 1^\circledast$ that factor $\kappa$ via a 1-cell $F$ and such that $\ee\circ \pt^\circledast \cong \id$, $\mm \circ \pt^\circledast\cong \id$, one has $\ee = F * \epsilon$ and $\mm = F * \eta$.
\end{lemma}
\begin{proof}
By the very definition of $\kappa$, the whiskering $\pt^\circledast * \kappa$ coincides with the counit\hyp{}unit composition $\pt^\circledast \circ 0^\circledast \xto{\epsilon} \id \xto{\eta} \pt^\circledast \circ 1^\circledast$. Thus, $m_{\firstblank}\circ e_{\firstblank}=(F * \eta)\circ (F * \epsilon)=F\circ \pt^\circledast*\kappa=\kappa$. The last statement is a simple formal consequence of the zig\hyp{}zag identities for the adjunctions $1^\circledast \adjunct{\eta}{} \pt^\circledast  \adjunct{}{\epsilon}0^\circledast$.
\end{proof}
\begin{remark}
Spelled out more explicitly, the above lemma shows that a normal factorization pre\hyp{}algebra $F$ functorially associates to a given $X\in \D^\due(I)$ a factorization 
\begin{equation}\label{func_fact}
\xymatrix{X_0 \ar@/^18pt/[rr] \ar[r]^{e_{X}}&FX\ar[r]^{m_{X}}&X_1}
\end{equation}
of the underlying diagram $X_0\to X_1$ of $X$. 
\end{remark}
In the following discussion we are going to show that such factorization can be made ``coherent'' via a morphism
\[
\Phi_F\colon \D^\due\longrightarrow \D^\tre
\]
such that $(0,2)^\circledast \circ \Phi_F=\id_{\D^\due}$, $1^\circledast \circ \Phi_F =F$ and $\dia_{\tre}(\Phi_FX)$ is the diagram in (\refbf{func_fact}).
\begin{notat}
Before giving the construction of $\Phi_F$, let us introduce a few more notation, this time regarding the category of functors $[\due\times\due,\due]$, that comes equipped with a commutative square of natural transformations as in the left diagram below (see also \cite[§\textbf{2.4}]{RW}), whose components are depicted on the right.
\[
\label{diamond_def}
\vcenter{
	\xymatrix{r \ar@{=>}[r]^\mu\ar@{=>}[d]^{\mu'}& h  \ar@{=>}[d]^\nu \\ v\ar@{=>}[r]_{\nu'} & l}
}\qquad\qquad
\vcenter{\xymatrix{
r(i,j)= i\land j \ar[d]^{\mu'_{ij}}\ar[r]^{\mu_{ij}} &  h(i,j)=j \ar[d]^{\nu_{ij}}\\
v(i,j)=i \ar[r]_{\nu'_{ij}} & l(i,j)=i\lor j 
}}
\]
where $\mu$, $\mu'$, $\nu$ and $\nu'$ obviously represent the two chains $i\land j \le i\le i\lor j$ and $i\land j \le j\le i\lor j$. Consider also the ``slit'' functor $d\colon \tre\times \due\to \due$ defined as follows:
\[
\begin{kodi}[xscale=2,yscale=1.2]
\foreach \x in {0,1,2} {
\foreach \y in {0,1}
  \node at (\y,-\x) (\x\y) {$(\x,\y)$};
}
\node at (2,-1) (label) {$d$};
\draw[densely dotted] (00.north west) -- 
					  (10.south west) -- 
					  (10.south east) -- 
					  (00.south east) -- 
					  (01.south east) -- 
					  (01.north east) -- cycle;
\draw[densely dotted] (11.north west) -- 
					  (11.north east) -- 
					  (21.south east) -- 
					  (20.south west) -- 
					  (20.north west) -- 
					  (21.north west) -- cycle;
\node at (3,0) (zero) {$0$};
\node at (3,-2) (uno) {$1$};
\draw[->] (01) to[bend left] (zero);
\draw[->] (21) to[bend right] (uno);
\mor 00 -> 01 -> 11 -> 21;
\mor * -> 10 -> 20 -> *;
\mor 10 -> 11;
\mor zero -> uno;
\end{kodi}
\]
\end{notat}
Having established this notation, we gather in the following statement some elementary facts about the above objects and arrows:
\begin{lemma}\label{properties_of_functors_among_ord}
In the above notation,
\begin{enumerate}[label=($\roman*$)]
\item there are adjunctions $l\adjunct{\varphi}{} \Delta\adjunct{}{\psi} r\colon \due\to \due\times\due$ 
(the counit of $l\dashv \Delta$ and the unit of $\Delta\dashv r$  are identities);
\item $v=\id_{\due}\times\pt\colon \due\times \due\to \due\times \uno=\due$;
\item $h=\pt\times\id_\due\colon \due\times \due\to \uno\times \due=\due$;
\item The identities $v\circ\Delta = h\circ\Delta =\id_\due$ hold;
\item $d\circ((0,1)\times \id_\due)=r$ and $d\circ((1,2)\times \id_\due)=l$;
\item $d\circ ((0,2)\times \id_{\due})=\id_{\due}\times\pt$;
\item $d\circ (1\times \id_{\due})=\id_{\due}$.
\end{enumerate}
\end{lemma}
\begin{definition}[coherent factorization and its pieces]\label{func_fact_def}
Given a pre-derivator $\D\colon \Dia^{op}\to \Cat$ and a normal factorization pre\hyp{}algebra $F\colon \D^\due\longrightarrow \D$, we define the following morphisms of derivators:
\begin{itemize}
\item $\Phi_F \coloneqq  F^\tre \circ d^\circledast\colon \D^\due\to \D^\tre$;
\item $F_l \coloneqq  F^\due \circ l^\circledast\colon \D^\due\to \D^\due$;
\item $F_r \coloneqq  F^\due \circ r^\circledast\colon \D^\due\to \D^\due$,
\end{itemize}
and the following natural transformations:
\begin{itemize}
\item $m_{e_{\firstblank}}\colon F\circ F_r=F\circ F^\due \circ r^\circledast\xto{FF^\due * \mu^\circledast} F\circ F^\due\circ  v^\circledast=F$;
\item $e_{m_{\firstblank}}\colon F=F\circ F^\due\circ  v^\circledast\xto{FF^\due * \nu^\circledast} F\circ F_l=F\circ F^\due\circ l^\circledast$.
\end{itemize}
\end{definition}
We are now ready to prove the announced properties of $\Phi_F$:
\begin{lemma}\label{prealgebra_induces_factorization}
In the above notation, the following statements hold true:
\begin{enumerate}
\item $(1,2)^\circledast \circ \Phi_F\cong F_l$ and $(0,1)^\circledast \circ\Phi_F\cong F_r$;
\item $(0,2)^\circledast \circ \Phi_F\cong \id_{\D^\due}$;
\item $1^\circledast \circ \Phi_F \cong F$.
\end{enumerate}
\end{lemma}
\begin{proof}
Since $\Phi_F$ is a morphism in $\PDer$, $(1,2)^\circledast\circ \Phi_F \cong F_l$ as a consequence of the chain of isomorphisms
\begin{align*}
(1,2)^\circledast\circ \Phi_F & =(1,2)^\circledast\circ F^\tre d^\circledast\\
& \cong F^\due \circ ((1,2)\times \id_\due)^\circledast \circ d^\circledast\\
&=F^\due \circ (d \circ((1,2)\times \id_\due))^\circledast\\
&=F^\due \circ l^\circledast=F_l
\end{align*}
where we used Lemma \refbf{properties_of_functors_among_ord}.($v$). This proves the first half of (1), the second half is completely analogous. Also, parts (2) and (3) follow similarly, using part ($vi$) and ($vii$) of \refbf{properties_of_functors_among_ord}, respectively.
\end{proof}
\begin{remark}
It is easy, though not needed in the following discussion, to define factorizations 
\[
\xymatrix{
	0^\circledast \ar[dr]\ar[rr]^e&& F\ar[rr]^m \ar[dr]|{FF^\due* \nu^\circledast}&& 1^\circledast \\
	& FF^\due r^\circledast \ar[ur]|{FF^\due * \mu^\circledast}&& FF^\due l^\circledast\ar[ur]
}
\]
and to show that these two triangles are, respectively, $\Phi_F(F_rX)$ and $\Phi_F(F_lX)$.
\end{remark}
We conclude the discussion with the following remark that shows how working with factorization pre\hyp{}algebras which are normal is not restrictive (this is completely analogous to \cite[§\textbf{2.2}]{Korostenski199357}):
\begin{remark}[normalization lemma]\label{normal_not_restrictive}
Given a factorization pre\hyp{}algebra $F\colon \D^\due\to \D$ with a fixed isomorphism $\gamma\colon F\circ \pt^\circledast \to \id_\D$ we can find another morphism $F'\colon \D^\due\to \D$ such that $F'\circ \pt^\circledast =\id_\D$ and $F'\cong F$. Indeed, given $I\in \Dia$, one defines $F'_I\colon \D^\due(I)\to \D(I)$ as follows: for an object $X\in \D^\due(I)$
\[
F'_I(X) \coloneqq \begin{cases}
Y&\text{if $X=\pt^*(Y)$;}\\
F_I(X)&\text{otherwise;}
\end{cases}
\]
while for a morphism $\phi\colon X\to X'$ in $\D^\due(I)$, 
\[
F'_I(\phi) \coloneqq \delta_{X'}\circ F_I(\phi)\circ \delta_{X}^{-1}\qquad\text{where}\qquad
\delta_X \coloneqq \begin{cases}
\gamma_Y&\text{if $X=\pt^*(Y)$;}\\
\id_{F_I(X)}&\text{otherwise.}
\end{cases}
\]
\end{remark}
\subsection{Eilenberg-Moore factorizations}\label{EM_subs}
In this subsection we are going to prove that, under very mild assumptions, a normal factorization pre\hyp{}algebra $F\colon \D^\due\to \D$ induces a \dfs $\F=(\mathbb E_F,\mathbb M_F)$ such that the functor $\Phi_F$ of \adef\refbf{func_fact_def} provides an inverse to the functor $\Psi_\F$ of \adef\refbf{def_hfs}.

Let us start defining the pre-derivators $\mathbb E_F$ and $\mathbb M_F$:

\begin{definition} \label{def_EF_MF}
In the same setting and with the same notations of \adef\refbf{func_fact_def}, we define two sub pre-derivators $\mathbb E_F$ and $\mathbb M_F\subseteq \D^{\due}$ where, for any $I\in \Dia$, 
\begin{align}
\mathbb{E}_F(I)&=\{X\in \D^{\due}(I)\mid F_{l}X\text{ is an iso}\}\notag\\ 
\mathbb{M}_F(I)&=\{Y\in \D^{\due}(I)\mid F_{r}Y\text{ is an iso}\}.
\end{align}
%($F_{r,I}, F_{l,I}$ denote the $I$\hyp{}component of $F_r$ and $F_l$ (\adef\refbf{func_fact_def}), and an object $X\in\D^{\due}(I)$ ``is an iso'' if its underlying diagram is an isomorphism or, equivalently, if $X\cong 0_{\circledbang,I}(Y)$ for some $Y\in\D(I)$).
\end{definition}
What allows us to prove that the pair $(\mathbb E_F,\mathbb M_F)$ is a \dfs is the rephrasing of the Eilenberg\hyp{}Moore condition.
\begin{definition}[eilenberg\hyp{}moore factorization]\label{def_EM_algebra}
A normal factorization pre\hyp{}algebra $F\colon \D^\due\to \D$ is said to be a \emph{Eilenberg\hyp{}Moore} (\emph{\textsc{em}}, for short) \emph{factorization} provided $F_r(X)\in \mathbb E_F(I)$ and  $F_l(X)\in \mathbb M_F(I)$, for any $I\in\Dia$ and $X\in \D^\due(I)$.
\end{definition}
We can now prove our awaited Theorem \textbf{III}:
\begin{proof}[Proof of Theorem \textbf{III}]
By Lemma \refbf{prealgebra_induces_factorization}, $\Phi_F$ takes values in $\D_\F\subseteq \D^\tre$ and $\Psi_\F\Phi_F=(0,2)^\circledast\Phi_F\cong \id_{\D^\due}$. This shows that $\Psi_\F$ is essentially surjective and full. Consider now $X\in \mathbb E_FI$ and $Y\in \mathbb M_FI$ and let us show that the map 
\[
\varphi_{X,Y}\colon\D^I(\uno)(X_1,Y_0)\longrightarrow \D^I(\due)(X,Y)
\] 
is an isomorphism. Indeed, $\Phi_F X\cong (0,1)_!X$ and $\Phi_F Y\cong (1,2)_*Y$, so that
\begin{align*}
\D(\uno)(X_1,Y_0) &\cong \D(\due)(X,1_*Y_0)\\
& \cong \D(\tre)((0,1)_!X,(1,2)_*Y)\\
& \cong \D(\tre)(\Phi_F X,\Phi_F Y)\\
& \twoheadrightarrow\D(\due)(\Psi_\F\Phi_F X,\Psi_\F\Phi_F Y)\\
& \cong \D(\due)(X,Y)
\end{align*}
showing that $\varphi_{X,Y}$ is surjective; it remains to show injectivity. Indeed, consider two morphisms $a,\, b\colon X_1\to Y_0$, such that 
$\varphi_{X,Y}(a)=\varphi_{X,Y}(b)$. This means that $\psi_Y\pt_\due^*a\varphi_X=\psi_Y\pt_\due^*b\varphi_X$ and so, in particular,
\[
F(\psi_Y)F(\pt^*a)F(\varphi_X)=F(\psi_Y)F(\pt^*b)F(\varphi_X).
\]
Now, $\psi_Y=\dia_{\due}(l^*Y)$ so $F(\psi_Y)=\dia_{\due}(F^{\due}l^*Y)=\dia_{\due}(F_lY)$ is an iso and, similarly, $F(\varphi_X)$ is an iso. 
Hence, we obtain that $a= F(\pt^*a)=F(\pt^* b)=b$. This proves conditions (1) and (3) of Lemma \refbf{easier_def_dfs}, while condition (2) easily follows by construction, thus $\F$ is a \textsc{dfs}.

On the other hand, Let $\F'$ be a \dfs and suppose that $\D$ is represented or that it is a stable derivator. In both settings we known that $\Psi_{\F'}\colon \D_{\F'}\to \D^{\due}$ is an equivalence. Fix a quasi-inverse $\Phi_{\F'}\colon \D^{\due}\to \D_{\F'}$ to $\Psi_{\F'}$ and let $F' \coloneqq 1^\circledast\circ\Phi_{\F'}$. One can show that $F'$ is an \textsc{em} factorization and that $\F'=(\mathbb E_{F'},\mathbb M_{F'})$.
\end{proof}
