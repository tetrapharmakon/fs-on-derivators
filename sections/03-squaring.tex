\section{Derivator factorization systems}\label{sec:squaring}

A {\em category of diagrams} is a full sub-2-category of the 2-category $\cat$ of small categories  that fulfills some closure properties, for which we refer to \cite[\adef\textbf{4.21}]{Moritz}. Let us just remark that every object $\mathbf{n}\in\cat$ belongs to any category $\Dia$ of diagrams.

For a given category of diagrams $\Dia$, a {\em pre-derivator} is  a 2-functor
\[
\D\colon \Dia^\opp\to \Cat
\]
A pre-derivator $\D$ is said to be {\em representable} if there is a category $\C$ such that $\D(I)=\C^{I}$ for any $I\in \Dia$, for any functor $u\colon J\to I$, the functor $\D(u)$ acts as 
\[
u^*\colon \C^I\to \C^J\qquad\text{such that}\qquad ( F\colon I\to \C)\mapsto (F\circ u\colon J\to I\to \C),
\]
and it acts on natural transformations in the obvious way. In the above situation we say that $\D$ is {\em represented by} $\C$, in symbols $\D=y(\C)$.

A pre-derivator $\D$ is a {\em derivator} if it satisfies a series of four axioms (Der1)--(Der4), for which we refer to \cite{Moritz}, as well as for the definitions of {\em pointed}, {\em strong}, and {\em stable} derivator. 

\medskip
Through this section let us fix the following minimal setting; from time to time we will to need work under stronger hypotheses (typically, we will assume that $\D$ is representable, or that it is a stable derivator):

\begin{setting}\label{setting_sec_3} 
Let $\Dia\subseteq \cat$ be a category of diagrams and we fix a pre-derivator
\[
\D\colon \Dia^\opp\longrightarrow \Cat
\]
that satisfies the following conditions for any $I\in \Dia$:
\begin{enumerate}
\item a morphism $\phi$ in $\D(I)$ is an isomorphism if and only if $\phi_i$ is an isomorphism in $\D(\uno)$ for any $i\in I$;
\item $\dia_\due(I)\colon \D^\due(I)\to \D(I)^\due$ is full and essentially surjective;
\item $\dia_\tre(I)\colon \D^\tre(I)\to \D(I)^\tre$ is full and essentially surjective.
\end{enumerate}
\end{setting}


If $\D=y(\C)$ for some category $\C$, then $\dia_I$ is an equivalence of categories for any $I\in \Dia$, so (1), (2) and (3) are always satisfied for this kind of pre-derivators. In fact, condition (1) is exactly (Der2), so in particular it is fulfilled by any derivator. In the language of \cite{Moritz}, condition (2) says that $\D$ is a strong pre-derivator. Finally, let us also remark that (1), (2) and (3) are satisfied by any stable derivator.

\subsection{The comonoid $\due$}\label{comonoid_due_subs}
Consider the \emph{point functor} $\pt\colon \due\to \uno$ that collapses the arrow category $\due$ to the point category $\uno$. This functor has both a right and a left adjoint choosing respectively the terminal and initial object of $\due$:
\[
\xymatrix@C=1.5cm{
**[l] 0\dashv \pt \dashv 1\colon \due\ar[r]|\pt & \ar@/_-1pc/@<3pt>[l]|{1}\ar@/_1pc/@<-3pt>[l]|{0} {\uno}
}
\]
where the left adjoint $0\colon \uno\to \due$ sends the unique object of $\uno$ to $0\in \due$, while the right adjoint $1\colon \uno\to \due$ sends the unique object of $\uno$ to $1\in \due$. 


\begin{remark}\label{two-comonoid}
Like every object of $\cat$, the category $\due$ has the structure of a comonoid, where the co-multiplication is give by the diagonal map $\Delta\colon \due\to \due\times\due$, and the counit by the point functor $\pt\colon \due \to \uno$ above. It is in fact easy to check by hand the co-associativity and co-unitality relations:
\[\begin{cases}
(\pt\times \id_\due)\circ\Delta=\id_\due=(\id_\due\times \pt)\circ\Delta\notag\\
(\Delta\times\id_\due)\circ\Delta=(\id_\due\times \Delta)\circ\Delta.
\end{cases}\]
\end{remark}

Applying $\D$ to these functors we obtain the following adjunctions and isomorphisms thereof:
\[
\xymatrix@C=1.5cm{
**[l] \pt_! \cong 1^* \dashv \pt^* \dashv 0^* \cong \pt_* \colon
\D(\uno)\ar[r]|{\pt^*}&\ar@/_-1pc/@<3pt>[l]|{0^*}\ar@/_1pc/@<-3pt>[l]|{1^*} {\D(\due)}
}
\]
It is in fact easy to see that $\pt_!\cong 1^*$, $1_*\cong \pt^*$, $\pt_*\cong 0^*$ and $0_!\cong \pt^*$. Furthermore, the functor $0\colon \uno \to \due$ is a \emph{sieve} and $1\colon \uno\to \due$ is a \emph{co-sieve}; thus whenever $\D$ is a pointed derivator, by \cite[Corollary \textbf{3.8}]{Moritz}, $0_*$ has a right adjoint $0^!\colon \D(\due)\to \D(\uno)$, while $1_!$ has a left adjoint $1^?\colon \D(\due)\to \D(\uno)$. Hence, we end up with the string of adjoint functors
\[
\xymatrix@M=3mm@R=2cm{
     \ar@<4.5em>[d]|{0^!}
    \ar@<-4.5em>[d]|{1^?}
	\D(\due)
	\ar@<-1.5em>[d]|{\begin{smallmatrix} {}\\ \pt_!\\ 1^* \\{}\end{smallmatrix}}
	 \ar@<1.5em>[d]|{\begin{smallmatrix} {}\\ \pt_* \\ 0^* \\{}\end{smallmatrix}}\\
	\ar@{^{(}->}[u]|{\begin{smallmatrix} {}\\ 0_! \\ \pt^*  \\ 1_*\\{}\end{smallmatrix}}
	\D(\uno)
	 \ar@{^{(}->}@<3em>[u]|{1_!}
	\ar@{^{(}->}@<-3em>[u]|{0_*}
}
\]
(the functors are depicted from left to right respecting the adjointness relation, and functors on the same arrow are canonically isomorphic). Notice that all functors $1_!,$ $0_!\cong 1_*$, and $0_*$ are fully faithful. This is obvious since intuitively these three functors send an object into its initial, identity, and terminal arrow respectively.
\begin{notat}\label{la-kappa}
As a consequence of the fact that $0_!$ is fully faithful, the composition $0_!0^* \xto{\epsilon} \id \xto{\eta} 1_*1^*\cong 0_!1^*$ is of the form $0_!\kappa$ for a unique $\kappa\colon 0^* \to 1^*$.
\end{notat}
\begin{lemma}\protect{\cite[\aprop\textbf{3.24}]{Moritz}}
Suppose $\D$ is a pointed derivator, let $C\colon \D(\due)\to \D(\uno)$, $F\colon \D(\due)\to \D(\uno)$, $\Sigma\colon \D(\uno)\to \D(\uno)$, and $\Omega\colon \D(\uno)\to \D(\uno)$ be respectively the \emph{cone}, \emph{fiber}, \emph{suspension} and \emph{loop} functors, as defined in \cite[§\textbf{3.3}]{Moritz}. Then, $C\cong 1^?$, $\Sigma\cong 1^?0_*$, $F\cong 0^!$, and $\Omega\cong 0^!1_!$.
\end{lemma}
We conclude this subsection with two technical lemmas, which apply in case $\D$ is a stable derivator, that will make our life easier in the rest of the section. 
\begin{lemma}[The standard triangle of a coherent morphism]\label{simple_tria_for_orth}
Suppose $\D$ is a stable derivator. Given $X\in \D(\due)$, there is a triangle of the form
$$X\xto{\varphi_X} \pt^*\pt_!X\longrightarrow 0_*C(X)\longrightarrow \Sigma X,$$
where $\varphi_X\colon X\to \pt^*\pt_!X$ is the unit of the adjunction $(\pt_!,\pt^*)$.
\end{lemma}
\begin{proof}
Complete $\varphi_X$ to a triangle as follows
\[
X\xto{\varphi_X} \pt^*\pt_!X\longrightarrow K\longrightarrow \Sigma X
\]
The underlying diagram of the above triangle has the following form
\[
\xymatrix{
X_0\ar[r]\ar[d]&X_1\ar@{=}[d]\ar[r]&C(X)\ar[r]\ar[d]&\Sigma X_0\ar[d]\\
X_1\ar@{=}[r]&X_1\ar[r]&0\ar[r]&\Sigma X_1
}
\]
It is then clear that $K\cong 0_*C(X)$.
\end{proof}
\begin{notat}
For each $i<j$ in $\{0,1,2\}$ we denote by $(i,j)$ the functor 
\[
(i,j)\colon \Delta^{\{0,1\}}\hookrightarrow \Delta^{\{i,j\}}\subset\Delta^{2};
\] 
this slightly unusual notation for the co-face maps $\{\delta_2^i\colon \due\to \tre \mid i=0,1,2\}$ is motivated by the belief that $X_{\delta^2_i}$ or $\delta_i^{2,*}X$ to denote the image of $X$ under $\delta_i^{2,*} \colon \D(\tre)\to \D(\due)$ are unreadable clutters confronted with the simpler $X_{(i,j)}$.
\end{notat}
\begin{lemma}[The standard triangle of a coherent 2-simpleX]\label{factorization_triangle}
Suppose $\D$ is a stable derivator. Given $X\in \D(\tre)$, there is a triangle of the form
\[
(1,2)_!X_{(0,1)}\to 1_!X_1\oplus (0,2)_!X_{(0,2)}\to X\to \Sigma (1,2)_!X_{(0,1)}.
\]
\end{lemma}
\begin{proof}
We adopt a construction similar to one contained in \cite{porta2015universal}. 
Let $\epsilon_1\colon 1_!1^*X\to X$ and $\epsilon_{(0,2)}\colon (0,2)_!(0,2)^* X \to X$ be the co-units of the respective adjunctions. These obviously give a  map 
\[
1_!X_1\oplus (0,2)_!X_{(0,2)}\xto{\smat{\epsilon_1 & \epsilon_{(0,2)}}} X
\]
that can be completed to a triangle
\[
K\to 1_!X_1\oplus (0,2)_!X_{(0,2)}\to X\to \Sigma K.
\]
Since $(1_!X_1\oplus (0,2)_!X_{(0,2)})_0 \cong (1_!X_1)_0\oplus ((0,2)_!X_{(0,2)})_0 \cong X_0$, then $K_0=0$. Given how the functor $(1,2)_!$ acts on objects, $K \cong (1,2)_! Y$ for some $Y$; we now aim to prove that such a $Y$ is necessarily isomorphic to $X_{(0,1)}$. For this, apply the functor $(1,2)^*$ to the above triangle, 
 to obtain the following triangle in $\D(\due)$:
\begin{equation}\label{bad_triangle}
((1,2)_!Y)_{(1,2)}\to (1_! X_1\oplus (1,2)_!X_{(0,2)})_{(1,2)}\to X_{(1,2)}\to \Sigma ((1,2)_!Y)_{(1,2)}.
\end{equation}
Notice that the obvious natural transformation $\gamma\colon (0,1) \Rightarrow (1,2)$, can be viewed as a composition of natural transformations in the following two ways:
\[
\xymatrix{
\due \ar[rr]|{(0,2)}^*!/u5pt/{\labelstyle\ \Downarrow\alpha}_*!/d5pt/{\labelstyle\ \Downarrow\beta} \ar@/_-25pt/[rr]|{(0,1)}\ar@/_25pt/[rr]|{(1,2)} && \tre & = & \due \ar@/_-25pt/[rr]|{(0,1)}\ar@/_25pt/[rr]|{(1,2)}\ar@{}[rr]^*!/u6pt/{\labelstyle\;\;\; \Downarrow\beta'}_*!/d6pt/{\labelstyle\;\;\; \Downarrow\alpha'}
\ar[r]|{\pt}&\uno\ar[r]|{1}& \tre
}
\]
giving us the upper left square in the following commuting diagram in $\D[1]$:
	\[
		\xymatrix@C=2cm{
		X_{(0,1)} \ar[r]^{\alpha_X^*}\ar[d]_{(\beta')_X^*}& X_{(0,2)}\ar[d]^{\beta_X^*}&(0,2)^*(0,2)_!X_{(0,2)}\ar[d]|{\cong}^{\beta^*_{(0,2)_!X_{(0,2)}}}\ar[l]|{\cong}_(.6){(0,2)^*\epsilon_{(0,2),X}}\ar@{}[ld]|{(\bullet)}\\
		\pt^*X_1 \ar[r]^{(\alpha')_X^*}& X_{(1,2)}&(1,2)^*(0,2)_!X_{(0,2)}\ar[l]^(.6){(1,2)^*\epsilon_{(0,2),X}}\\
		\pt^*(1_!X_1)_1\ar[u]|\cong^{\pt^*1^*(\epsilon_{1,X})}\ar[r]|\cong_{(\alpha')_{1_!X_1}^*}&(1,2)^*1_!X_1\ar[u]_{(1,2)^*\epsilon_{1,X}}\ar@{}[lu]|{(\bullet\bullet)}
		}
	\]
The commutative squares marked by ($\bullet$) and ($\bullet\bullet$) tell us that the triangle in \eqref{bad_triangle} is isomorphic to a triangle of the form:
$$\xymatrix@C=2cm{
Y\ar[r]& \pt^* X_1\oplus X_{(0,2)}\ar[r]^(.6){((\alpha')^*,\beta^*)}& X_{(1,2)}\ar[r]& \Sigma Y,}$$
while the third commutative square shows that the following composition is trivial:	
\[
\xymatrix@C=2cm{ X_{(0,1)} \ar[r]^(.4){[(\beta')^*,-\alpha^*]^t} & 0_!X_1 \oplus X_{(0,2)} \ar[r]^(.6){[(\alpha')^*,\beta^*]} & X_{(1,2)}}
\]
We obtain a map $\varphi\colon X_{(0,1)}\to Y$ making the following diagram commutative:
\[
\xymatrix{
X_{(0,1)}\ar[dr]\ar@{.>}[d]|{\varphi}\\
Y\ar[r]& \pt^* X_1\oplus X_{(0,2)}\ar[r]& X_{(1,2)}\ar[r]& \Sigma Y
}
\]
To conclude our proof, it is enough to show that $\varphi$ is an isomorphism. For this, it is enough to show that $\varphi_0\colon X_0\to Y_0$ and $\varphi_1\colon X_1\to Y_1$ are isomorphisms in $\D(\uno)$. 
But in fact, applying $0^*$ and $1^*\colon \D(\due)\to \D(\uno)$ to the above diagram, we get the following diagrams in $\D(\uno)$:
\[
\xymatrix@C=20pt@R=20pt{
X_0\ar[d]\ar[dr]&&&       &           X_1\ar[d]\ar[rd]\\
Y_0\ar[r]& X_1\oplus X_0\ar[r]& X_1\ar[r]& \Sigma Y_0        &       Y_1\ar[r]& X_1\oplus X_2\ar[r]& X_2\ar[r]& \Sigma X_1
}
\]
respectively. This shows that $\varphi_0$ is a morphism that factors the kernel $X_0\to X_1\oplus X_0$ of the morphism $X_1\oplus X_0\to X_1$ through the kernel $Y_0\to X_1\oplus X_0$ of the same map. Hence, $\varphi_0$ is an isomorphism. A completely analogous argument shows that $\varphi_1$ is an isomorphism.
\end{proof}





\subsection{Coherent orthogonality}\label{Subs_Coh_Ort}
The objects of the category $\D(\due)$ can be thought of as ``coherent morphisms'' of $\D(\uno)$ (as opposed to the ``incoherent morphisms'', which are the objects of $\D(\uno)^{\due}$); in general the underlying diagram functor $\dia_\due \colon \D(\due)\to \D(\uno)^\due$ has no property whatsoever that ensures that a coherent diagram $X\in\D(\due)$ leaves a faithful image in its associated incoherent diagram $\dia_\due(X)$ (however, in our Setting \refbf{setting_sec_3}, $\dia_\due$ is at least full and essentially surjective).

In this subsection we are introducing a notion of \emph{coherent orthogonality} for a pair of objects in $\D(\due)$ that takes into account the richer structure of coherent diagrams. Indeed, let $X, Y\in\D(\due)$ and consider the unit $\varphi_X\colon X\to \pt^*\pt_!X$ of the adjunction $(\pt_!,\pt^*)$. Applying $\D(\due)(-,Y)$, and recalling that $\pt_!X\cong X_1$ and $\pt_*Y\cong Y_0$, we obtain a natural morphism
\[
\xymatrix@R=0pt{\D(\uno)(X_1,Y_0)\ar[r]^{\varphi_{X,Y}}&\D(\due)(X,Y)\\
(\pt_!X\xto{a} \pt_*Y)\ar@{|->}[r]&(X\xto{\varphi_X} \pt^*\pt_!X\xto{\pt^*a} \pt^*\pt_*Y\xto{\psi_Y} Y),}
\]
where $\psi_Y\colon\pt^*\pt_*Y\to Y$ is the counit of the adjunction $(\pt^*,\pt_*)$.

\begin{definition}[coherent orthogonality]\label{def_c_ort}
Given $X,Y\in \D(\due)$, we say that $X$ is \emph{left coherently orthogonal} to $Y$ (while $Y$ is \emph{right coherently orthogonal} to $X$), in symbols $X\corth Y$, if the map ${\varphi_{X,Y}}\colon\D(\uno)(X_1,Y_0)\to \D(\due)(X,Y)$ is an isomorphism.\\
Given $\mathcal{X}\subseteq \D(\due)$, we let
\begin{gather}
\mathcal{X}^{\corth } \coloneqq \{Y\in \D{(\due)}:X\corth Y, \ \forall X\in \mathcal{X}\}\notag\\
{}^{\corth }\mathcal{X} \coloneqq \{X\in \D{(\due)}:X\corth Y, \ \forall Y\in \mathcal{X}\}.\notag
\end{gather}
\end{definition}

Our first observation about coherent orthogonality is that, in case $\D$ is representable, we recover the classical notion of orthogonality of morphisms:
\begin{lemma}\label{corth=orth}
In Setting \refbf{setting_sec_3},  consider $X$ and $Y\in \D(\due)$. If $X\corth Y$ then for any commutative diagram 
\[
\xymatrix{
X_0\ar[d]_{\dia_\due X}\ar[r]^{\phi_0}&Y_0\ar[d]^{\dia_\due Y}\\
X_1\ar[r]_{\phi_1}\ar@{.>}[ur]|d&Y_1
}
\]
there is a $d\colon X_1\to Y_0$ such that $\phi_0=d\circ \dia_\due X $ and $\phi_1=\dia_\due Y \circ d$. Furthermore, if $\D=y(\C)$ is representable,  then $X\corth Y$ if and only if, in the above diagram, there is a unique arrow $d\colon X_1\to Y_0$ such that $\phi_0=d\circ X$ and $\phi_1=Y\circ d$.
\end{lemma}
\begin{proof}
By condition (2) in Setting \refbf{setting_sec_3}, a morphism $(\phi_0,\phi_1)\colon \dia_\due X \to \dia_\due Y $ can be lifted to a morphism $\phi\colon X\to Y$ such that $\dia_\due \phi =(\phi_0,\phi_1)$; if furthermore $\D$ is representable, than this lifting is unique. Now, $X\corth Y$ if and only if, given $\phi\colon X\to Y$ there is a unique morphism $\widetilde d\colon X_1\to Y_0$ such that $\varphi_{X,Y}(\widetilde d)=\phi$. This means that, letting $d \coloneqq \dia_\due(\widetilde d)$, we get a commutative diagram
\[
\xymatrix{
X_0\ar[d]_{\dia_\due X }\ar[r]^{\dia_\due X }&X_1\ar@{=}[d]\ar[r]^d&Y_0\ar@{=}[d]\ar@{=}[r]&Y_0\ar[d]^{\dia_\due Y }\\
X_1\ar@{=}[r]&X_1\ar[r]_d&Y_0\ar[r]_{\dia_\due  Y }&Y_1
}
\]
such that the composition of the top row is $\phi_0$ and that of the bottom row is $\phi_1$.
\end{proof}

The second thing we would like to point out is that, in case $\D$ is a stable derivator, so that $\D(\uno)$ is canonically a triangulated category,
then coherent orthogonality is equivalent to the homotopy orthogonality introduced in Definition \refbf{wobbly}:
\begin{lemma}\label{coherent_orth_is_wobbly}
Suppose $\D$ is a stable derivator and let $X,$ $Y\in \D(\due)$. Then, $X\corth Y$ if and only if $\dia_{\due}X\horth \dia_{\due}Y$.
\end{lemma}
\begin{proof}
Consider the triangle $X\to \pt^*\pt_!X\to 0_*C(X)\to \Sigma X$,  given by Lemma \refbf{simple_tria_for_orth}. Now apply $\D(\due)(-,Y)$ to this triangle to get the following long exact sequence:
\begin{align*}
\label{ort_ses}
\cdots &\to \D(\uno)(C(X),\Sigma^{-1}C(Y))\to \D(\uno)(X_1,Y_0)\to \D(\due)(X,Y)\to \\
\notag &\to\D(\uno)(C(X),C(Y))\to \D(\uno)(X_1,\Sigma Y_0)\to\cdots 
\end{align*}
By definition, $X\corth Y$ if and only if the map $\D(\uno)(X_1,Y_0)\to \D(\due)(X,Y)$ is bijective, but this map is injective if and only if  
\[
\D(\uno)(C(X),\Sigma^{-1}C(Y))\to \D(\uno)(X_1,Y_0)
\] 
is trivial (which is condition \refbf{wobbly}.\textsc{ho}1 for $(\dia_{\due}X,\dia_{\due}Y)$), while it is surjective if and only if the map 
\[
\D(\uno)(C(X),C(Y))\to \D(\uno)(X_1,\Sigma Y_0)
\] 
is injective (which is condition \refbf{wobbly}.\textsc{ho}2 for $(\dia_{\due}X,\dia_{\due}Y)$).
\end{proof}

Another characterization of coherent orthogonality, this time for classes, in a stable derivator $\D$, can be given using the composition functor $(0,2)^*\colon \D(\tre)\to \D(\due)$:

\begin{lemma}\label{level_wise_ff}
Suppose $\D$ is a stable derivator. The following are equivalent for a pair $\F=(\E,\M)$ of subclasses of $\D(\due)$:
\begin{enumerate}
\item $\E\corth \M$;
\item letting $\D_\F(\uno)\subseteq \D(\tre)$ be the full subcategory of those $X\in \D(\tre)$ such that $(0,1)^*X\in \E$ and $(1,2)^*X\in\M$, the restriction 
\[
\Psi:=(0,2)^*\restriction_{\D_\F(\uno)}\colon \D_\F(\uno)\to \D(\due)
\] 
is fully faithful. 
\end{enumerate}
\end{lemma}
\begin{proof}
Given $X,\, Y\in \D(\tre)$, by Lemma \refbf{factorization_triangle} there is a triangle
\[
(1,2)_!X_{(0,1)}\to 1_!X_1\oplus (0,2)_!X_{(0,2)}\to X\to \Sigma (1,2)_!X_{(0,1)}.
\]
Applying $\D(\tre)(-,Y)$ to this triangle we get a long exact sequence:
\begin{align*}
\cdots&\to \D(\due)(\Sigma X_{(0,2)},Y_{(0,2)})\oplus\D(\uno)(\Sigma X_1,Y_1)\xto{(*)} \D(\due)(\Sigma X_{(0,1)},Y_{(1,2)})\to\notag\\
&\to\D(\tre)(X,Y)\to\D(\due)(X_{(0,2)},Y_{(0,2)})\oplus\D(\uno)(X_1,Y_1)\to\notag\\
&\xto{(*)} \D(\due)(X_{(0,1)},Y_{(1,2)})\to \cdots%\label{long_exact_factorization_sequence}
\end{align*}
If $X,\, Y\in\D_{\F}(\uno)$, then $X_{(0,1)},\, \Sigma X_{(0,1)}\in \E$ and $Y_{(1,2)}\in\M$. Thus, the following canonical maps are isomorphisms:
\begin{gather*}
\D(\uno)(X_1,Y_1)\xto{\cong}\D(\due)(X_{(0,1)},Y_{(1,2)})\\
\D(\uno)(\Sigma X_1,Y_1)\xto{\cong}\D(\due)(\Sigma X_{(0,1)},Y_{(1,2)})
\end{gather*}
showing that the two maps marked by $(*)$ in the above long exact sequence are (split) surjections. This shows that the natural map
\[
\D(\tre)(X,Y)\xto{\cong} \D(\due)(X_{(0,2)},Y_{(0,2)})
\]
is an isomorphisms, that is, $\Psi$ is fully faithful.
\end{proof}



We omit the proof of the following easy result

\begin{proposition}\label{ortho_to_self}
The following conditions are equivalent, for $X\in\D(\due)$.
\begin{enumerate}
	\item $X\corth X$;
	\item $X$ is an isomorphism (\ie $\dia_\due(X)$ is an isomorphism in $\D(\uno)$);% or, equivalently, $X$ belongs to the essential image of $0_!\cong 1_*\colon\D(I)\cong \D(I\times[0])\to \D(I\times [1])$);
	\item $X\corth \D(\due)$;
	\item $\D(\due)\corth X$.
\end{enumerate}
\end{proposition}


\begin{definition}[derivator pre-factorization systems]\label{def_phfs}
Denote by $\D^{\due}$ the shifted pre-derivator $\D^{\due}(I) \coloneqq \D(\due\times I)$. 
Let $\mathbb E$ and $\mathbb M$ be two sub pre-derivators of $\D^{\due}$.
For any $I\in \Dia$, let $\E_I \coloneqq \mathbb E(I)$, $\M_I \coloneqq \mathbb M(I)$ and $\F_I \coloneqq (\E_I,\M_I)$.%, identifying $\E_I$ and $\M_I$ with two full subcategories of $\D^\due(I)$.

The pair $\F \coloneqq (\mathbb E,\mathbb M)$ is a \emph{derivator pre-factorization system} (\cpfs for short) if $\E_I^{\corth}=\M_I$ and ${}^{\corth }\M_I=\E_I$,  for any $I\in\Dia$.
\end{definition}

The following lemma, whose proof is an easy consequence of Lemma \refbf{corth=orth}, describes the {\cpfs}s on a representable pre-derivator.

\begin{lemma}\label{dpfs=pfs_if_discrete}
Suppose that $\D=y(\C)$ is representable and let $\F=(\mathbb E,\mathbb M)$ be a pair of sub pre-derivators of $\D^\due$. Then, $\F$ is a \cpfs if and only if each $\F_I$ is an orthogonal pre-factorization system (see, for example, \cite[§\textbf{1}]{riehl2008factorization}).
\end{lemma}

Our next task is to describe the {\cpfs}s on a stable derivator $\D$ in terms of {\phfs}s on its images. Before that, we prove the following lemma giving some useful closure properties of {\cpfs}s.

%Notice that, given a \cpfs $\F=(\mathbb E,\mathbb M)$, the derivators $\mathbb E$ and $\mathbb M$ are pointed, and the morphism $\iota_e$ and $\iota_m$ do preserve zero-objects. To see this, let $I\in\Dia$ and notice that $0\in \D^{[1]}(I)$ is left and right coherently orthogonal to any other object $X\in \D^{[1]}(I)$:
%\[
%\D(I)(0,X_0)\xto{\cong} \D^{[1]}(I)(0,X) \quad\text{and}\quad \D(I)(X_1,0)\xto{\cong} \D^{[1]}(I)(X,0)
%\]
%In fact, much more is true:

\begin{lemma}\label{cfs_limit_colimit}
Suppose $\D$ is a derivator and let $\F=(\mathbb E,\mathbb M)$ be a \cpfs on $\D$. Given a functor $u\colon J\to I$ in $\Dia$, 
\begin{enumerate}
\item if $X\in \mathbb E(J)$, then $u_!X\in \mathbb E(I)$;
\item if $X\in \mathbb M(J)$, then $u_*X\in \mathbb M(I)$.
\end{enumerate}
\end{lemma}
\begin{proof}
Let $X\in \mathbb E(J)$ and $Y\in \mathbb M(I)$, then
\begin{align*}\D^\due(I)(u_!X,Y)&\cong \D^\due(J)(X,u^*Y)\\
&\cong \D(J)(X_1,(u^*Y)_0)\\
&\cong \D(J)(X_1,u^*(Y_0))\\
&\cong \D(I)(u_!(X_1),Y_0)\\
&\cong \D(I)((u_!X)_1,Y_0)
\end{align*}
where we used that $1^*$ commutes with Kan extensions, see \cite[\aprop\textbf{2.6}]{Moritz}. This shows that $u_!x{\corth }Y$ for any $Y\in\mathbb M(I)$, and so $u_!X\in \mathbb E(I)$. This proves (1), the proof of part (2) is completely analogous.
\end{proof}


%\begin{proposition}\label{preco-are-homo}
%Suppose $\D$ is a stable derivator and let $\F=(\mathbb E,\mathbb M)$ be a pair of sub pre-derivators of $\D^\due$. Given $\class{X}\subseteq \D^\due(I)$ we denote $\overline{\class{X}}$ the isomorphism\hyp{}closure of the class $\dia_\due(\class{X})\subseteq \D(I)^\due$. Then the following are equivalent:
%\begin{enumerate}
%\item $\F$ is a \cpfs;
%\item $\overline\F_I \coloneqq (\overline\E_I,\overline\M_I)$ is a \phfs in the triangulated category $\D(I)$, for any $I\in\Dia$. 
%\end{enumerate}
%\end{proposition}
%\begin{proof}
%By Lemma \refbf{coherent_orth_is_wobbly}, $(\overline\E_I)^{\horth}=\overline\M_I$ and ${}^{\horth}(\overline\M_I)=\overline\E_I$ if and only if $(\E_I)^{\corth}=\M_I$ and ${}^{\corth}(\M_I)=\E_I$. Then clearly (2) implies (1). For the converse, it is enough to show that $\overline\E_I$ is closed under suspensions, which is a consequence of Lemma \refbf{cfs_limit_colimit}.
%\end{proof}





\color{black}
Let $\D$ be a prederivator, $\F=(\mathbb E,\mathbb M)$ a pair of sub pre-derivators of $\D^\due$ and let 
$\D_\F\colon \Dia^\opp\to \Cat$
be a pre-derivator such that $\D_{\F}(I)\subseteq \D(I\times \tre)$ is the full subcategory spanned by those $X\in  \D(I\times \tre)$ such that $X_{(0,1)}\in \E_I$ and $X_{(1,2)}\in \M_I$. Denote by 
\begin{equation}\label{composition_morphism}
\Psi_\F\colon \D_\F\longrightarrow \D^{\due}
\end{equation}
the restriction of the morphism of derivators $(0,2)^\circledast\colon\D^{\tre}\to \D^{\due}$. In case $\D$ is representable, it is known (and not difficult to verify by hand) that the $\Psi_\F$ is fully faithful. 

\begin{definition}[Choric {\dpfs}]
In the above notation, $\F$ is said to be \emph{choric} if $\Psi_{\F}$ is fully faithful. 
\end{definition}

We do not known of any example of a non-choric \dpfs. In fact, for stable (and representable) derivators, any \dpfs is automatically choric. Furthermore, in the stable setting,  it is equivalent to specify a \dpfs and a ``compatible family'' of {\phfs}s: 
\color{black}
\begin{theorem}\label{stable_equv_orth_pre}
Suppose $\D$ is a stable derivator. Given $\class{X}\subseteq \D^\due(I)$ we denote $\overline{\class{X}}$ the isomorphism\hyp{}closure of the class $\dia_\due(\class{X})\subseteq \D(I)^\due$. The following are equivalent for a pair of sub pre-derivators $\F=(\mathbb E,\mathbb M)$  of $\D^{\due}$:
\begin{enumerate}%[label=(\roman)*]
\item $\F$ is a \cpfs;
\item \color{black}$\F$ is a choric \cpfs;\color{black}
\item $\overline\F_I=(\overline\E_I,\overline\M_I)$ is a \phfs in $\D(I)$ for any $I\in\Dia$.
\end{enumerate}
\end{theorem}
\begin{proof}
The implication ``(2)$\Rightarrow$(1)'' is trivial while ``(1)$\Rightarrow$(2)'' is implied by Lemma \ref{level_wise_ff}. 
For the equivalence ``(3)$\Leftrightarrow$(1)'', we obtain by  Lemma \refbf{coherent_orth_is_wobbly}, that $(\overline\E_I)^{\horth}=\overline\M_I$ and ${}^{\horth}(\overline\M_I)=\overline\E_I$ if and only if $(\E_I)^{\corth}=\M_I$ and ${}^{\corth}(\M_I)=\E_I$. Then clearly (3) implies (1). For the converse, it is enough to show that $\overline\E_I$ is closed under suspensions, which is a consequence of Lemma \refbf{cfs_limit_colimit}.
\end{proof}









\subsection{Derivator factorization systems} 




We are now going to give the definition of a derivator factorization system. Roughly speaking, this should be a \cpfs $\F$ such that any map is ``coherently $\F$-crumbled''; in the following definition we translate this idea in the language of derivators.


\begin{definition}[derivator factorization systems]\label{def_hfs}
Let $\F=(\mathbb E,\mathbb M)$ be a pair of sub pre-derivators of $\D^\due$ and let $\Psi_\F\colon \D_\F\longrightarrow \D^{\due}$ be the morphism defined in \eqref{composition_morphism}. We say that $\F$ is a \emph{derivator factorization system} (for short, \dfs) if it is a \cpfs and if $\Psi_\F(I)$ is essentially surjective for any $I\in \Dia$.
\end{definition}

Let us give an interpretation of the above definition in case $\D$ is representable:

\begin{lemma}\label{weak_and_cancellation}
In Setting \refbf{setting_sec_3}, let $\F=(\mathbb E,\mathbb M)$ be a pair of sub pre-derivators of $\D^\due$. If $\F$ is a \dfs, then $\overline \F_I \coloneqq (\overline\E_I,\overline \M_I)$ is a weak factorization system (see, for example, \cite[§\textbf{2}]{riehl2008factorization}) on $\D(I)$, for any $I\in \Dia$. Also, $\overline \F_I$ has the following cancellation properties:
\begin{enumerate}
\item given a composition $g\circ f\in \D(I)^\due$, if $g\circ f$ and $f\in \overline\E_I$, then $g\in \overline\E_I$;
\item given a composition $g\circ f\in \D(I)^\due$, if $g\circ f$ and $g\in \overline\M_I$, then $f\in \overline\M_I$.
\end{enumerate}
If $\D=y(\C)$ is representable, then $\F$ is a \dfs if and only if each $\F_I$ is an orthogonal factorization system.
\end{lemma}
\begin{proof}
By Lemma \refbf{corth=orth} it is clear that the two classes $\overline\E_I$ and $\overline \M_I$ are weakly orthogonal and, using the essential surjectivity of $\dia_\tre(I)$, it is not difficult to show that any morphism in $\D(I)$ is $\overline\F_I$-crumbled. To show that $(\overline\E_I,\overline \M_I)$ is a weak factorization system it enough to show that $\overline\E_I$ and $\overline \M_I$ are closed under retracts, which is an easy exercise. 

Let us now verify the cancellation properties (1) and (2). As in the proof of Lemma \refbf{factorization_triangle}, consider the unique possible natural transformations $\gamma\colon (0,1)\Rightarrow (1,2)$ and $\beta\colon (0,2)\Rightarrow (1,2)$. Let $Z\in \D^\tre(I)$ be an object such that 
\[
\dia_{\tre}Z\cong[ \cdot \xto{f}\cdot\xto{g}\cdot]\in \D(I)^\tre.
\] 
Suppose that $Z_{(0,2)}$ and $Z_{(0,1)}\in \mathbb E(I)$, and let $Y\in \mathbb M(I)$. Given a morphism $\phi\colon Z_{(1,2)}\to Y$, there exists a unique morphism $d\colon Z_2\to Y_0$ such that $\varphi_{Z_{(0,2)},Y}(d)=\phi\circ \beta^*_Z$. Then, $\varphi_{Z_{0,1}, Y}(d\circ g)=\phi\circ\gamma^*_Z=\varphi_{Z_{0,1}, Y}(\phi_0)$, so that $d\circ g=\phi_0$. This shows that $g=\dia_{\due}Z_{(1,2)}$ is weakly orthogonal to $\dia_\due Y$ for any $Y\in \M_I$, so $g\in \overline \E_I$. This proves (1), the proof of (2) is completely analogous. 

The last statement follows by Lemma \refbf{dpfs=pfs_if_discrete} and the fact that saying that $\Psi_{\F}(I)$ is essentially surjective is equivalent to say that any morphism in $\D(I)$ is $\overline \F_I$-crumbled when $\D$ is represented. 
\end{proof}

Before analyzing \textsc{dfs}s in the context of stable derivators, let us give the following reformulation of their definition:

\begin{lemma}\label{easier_def_dfs}
In Setting \refbf{setting_sec_3}, let $\F=(\mathbb E,\mathbb M)$ be a pair of sub-2-functors of $\D^\due$. Then, $\F$ is a \textsc{dfs} if and only if the following statements hold true for any $I\in \Dia$:
\begin{enumerate}
\item $\mathbb E(I)\corth \mathbb M(I)$ (that is $\mathbb E(I)\subseteq {}^{\corth} \mathbb M(I)$ or, equivalently, $\mathbb M(I)\subseteq  \mathbb E(I)^{\corth}$);
\item $\mathbb E(I)$ and $\mathbb M(I)$ are closed under isomorphisms in $\D^\due(I)$;% and they both contain the essential image of $\pt^*(I)\colon \D(I)\to \D^\due(I)$;
\item $\Psi_\F$ is essentially surjective.
\end{enumerate}
\end{lemma}
\begin{proof}
It is trivial to verify that if $\F$ is a \textsc{dfs} then it satisfies (1), (2) and (3). On the other hand, let $I\in \Dia$, suppose (1), (2) and (3) are satisfied and let us prove that $\mathbb E(I)\supseteq {}^{\corth} \mathbb M(I)$ (the proof that $\mathbb M(I)\supseteq  \mathbb E(I)^{\corth}$ is completely analogous). Indeed, let $X\in{}^{\corth} \mathbb M(I)$ and consider an object $\widetilde X\in \D_\F(I)$ such that $\Psi_\F(\widetilde X)\cong X$. By construction, $(0,1)^*\widetilde X\in \mathbb E(I)\subseteq {}^{\corth}\mathbb M(I)$ and $(0,2)^*\widetilde X\cong X\in {}^{\corth}\mathbb M(I)$; by the same argument used in the proof of the cancellation properties in Lemma \refbf{weak_and_cancellation}, one verifies that $(1,2)^*\widetilde X\in {}^{\corth}\mathbb M(I)$, but then $(1,2)^*\widetilde X\in {}^{\corth}\mathbb M(I)\cap \mathbb M(I)$, showing that $\dia_\due((1,2)^*\widetilde X)$ is an isomorphism. As a consequence, $\dia_\due(X)\cong \dia_\due((0,1)^*\widetilde X)$ and, using Setting \refbf{setting_sec_3}(1,2), this implies $X\cong (0,1)^*\widetilde X\in \mathbb E(I)$.
\end{proof}


We close the subsection showing that the bijections of \athm \ref{stable_equv_orth_pre} restrict to {\dfs}s:


\begin{theorem}\label{stable_equv_orth}
Suppose $\D$ is a stable derivator. The following are equivalent for a pair of sub pre-derivators $\F=(\mathbb E,\mathbb M)$  of $\D^{\due}$:
\begin{enumerate}%[label=(\roman)*]
\item $\F$ is a \dfs;
\item \color{black}$\F$ is a choric \dfs;\color{black}
\item $\Psi_\F\colon \D_\F\to \D^{\due}$ is an equivalence;
\item $\overline\F_I=(\overline\E_I,\overline\M_I)$ is a \hfs in $\D(I)$ for any $I\in\Dia$.
\end{enumerate}
\end{theorem}
\begin{proof}
The equivalence ``(1)$\Leftrightarrow$(2)'' follows by \athm \ref{stable_equv_orth_pre}, while the equivalence ``(2)$\Leftrightarrow$(3)'' easily follows from the definitions. The implication ``(1)$\Rightarrow$(4)'' follows by Lemma \refbf{coherent_orth_is_wobbly} and the fact that $\dia_{I,\due}\colon \D^I(\due)\to \D(I)^{\due}$ is full and essentially surjective for any $I\in \Dia$. Finally, to prove the implication ``(4)$\Rightarrow$(1)'' notice that, by Lemma \refbf{coherent_orth_is_wobbly}, we know that $\F$ is a  \cpfs, let us show that each $\Psi_{\F}(I)$ is essentially surjective. For this, remember that the diagram functor
\[
\dia_{I,\tre}\colon \D(I\times \tre)\to \D(I)^{\tre}
\]
is full and essentially surjective. Let $X\in \D^I(\due)$ and choose two composable morphisms $\bar X_e\in \bar \E_I$ and $\bar X_m\in \bar \M_I$ such that $\dia_{I,\due}X=\bar X_m\circ \bar X_e$. We obtain a morphism $(f_0,f_1,f_2)\colon (X_0\to X_0\to X_1)\to (X_0\to (X_e(1)=X_m(0))\to X_1)$ in $\D(I)^{\tre}$ as in the following diagram:
\[
\xymatrix{
X_0\ar@{=}[d]\ar[r]^(.3){\id_{X_0}}&X_0= X_e(0)\ar[d]^{\bar X_e}\\
X_0\ar[r]^(.3){\bar X_e}\ar[d]_{\dia_{I,\due}X}&X_e(1)=X_m(0)\ar[d]^{\bar X_m}\\
X_1\ar[r]^(.3){\id_{X_1}}&X_1=X_m(1)
}
\]
Since $\dia_{I,\tre}$ is full and essentially surjective, we can lift the above diagram to a morphism $f\colon (1,2)_*X\to F$, where the underlying diagram of $F$ is exactly $(X_0\to (X_e(1)=X_m(0))\to X_1)$, so $F\in \D_\F(I)$. To conclude, one should just prove that $F_{(0,2)}\cong X$, but in fact $f_{(0,2)}\colon (0,2)^*(1,2)_*X(\cong X)\to F_{(0,2)}$ is an isomorphism. To see this just notice that $f_{(0,2)}$ is an isomorphism if and only if $0^*f_{(0,2)}$ and $1^*f_{(0,2)}$ are isomorphisms but, by construction, $0^*f_{(0,2)}=f_0=\id_{X_0}$ and $1^*f_{(0,2)}=f_2=\id_{X_2}$ are clearly isomorphisms.\qedhere
\end{proof}



\subsection{The relation with \textsc{fs} in $\infty$-categories} \label{infty_cat_fs}
Recall that an \emph{$\infty$-category} \cite{HTT} or \emph{quasi-category} \cite{joyal2008notes} is defined as a simplicial set $X$ in which every \emph{inner horn} $\Lambda^k[n] \to X$ has an extension $\Delta^n \to X$ along the inclusion $\Lambda^k[n]\to \Delta^n$. 

These liftings take care of the complicated ladder of coherence conditions for compositions in a $(\infty,1)$-category, as well as of the invertibility of all cells in dimension $k\ge 2$, making quasicategories into a flexible model to re-enact classical categorical constructions inside $X$ (co/limits, Kan extensions) and outside $X$ (adjoints, monads, monoidal structures). We refer to the sources \cite{HTT} or \cite{joyal2008notes} for a general background and the terminology and notation that we borrow.

Let's fix through this subsection an $\infty$-category $\iC$. In this subsection we are going to recall how to associate to $\iC$ a pre-derivator $\D_\iC$ and then, under suitable hypotheses, establish a natural bijection between the family of \textsc{fs}s in $\iC$ (in the sense of \cite{joyal2008notes}) and a family of {\dfs}s on $\D_\C$ called ``maximal {\dfs}s''. 
%
% Let us just recall that, given $a, b\in \iC_0$ we let
% \[
% \iC(a,b) \coloneqq \{f\colon \Delta^{1}\to \iC:f_0=a\ \text{ and }f_1=b\}\subseteq \iC^{\Delta^{1}}.
% \]
% Then, given a third object $c\in \iC_0$, there is a composition law 
% \[
% \xymatrix@R=0pt{\iC(b,c)\times \iC(a,b)\ar[r]& \iC(a,c)\\
% (g,f)\ar@{|->}[r] &g\circ f}
% \]
% which is well-defined and associative up to homotopy. We can now recall the definition of \textsc{fs} in a $\infty$-category.
\begin{definition}[squares and fillers]
Any two edges $f,\, g\colon \Delta^{1}\to \iC$ can be considered as elements in $(\iC^{\Delta^{1}})_0$. By the definition of internal hom in a presheaf topos, an element $q\in \iC^{\Delta^{1}}(f,g)$ is a square $q\colon \Delta^{1}\times \Delta^{1} \to \iC$ such that $q\restriction_{\Delta^{\{0\}}\times \Delta^{1}}=f$ and $q\restriction_{\Delta^{\{1\}}\times \Delta^{1}}=g$. We define a \emph{filler} for $q\in \iC^{\Delta^{1}}(f,g)$ to be an element $s\in \iC(f_1,g_0)$ such that 
\begin{enumerate}
\item $q\restriction_{\Delta^{1}\times \Delta^{\{0\}}}$ is homotopic to $d\circ f$;
\item $q\restriction_{\Delta^{1}\times \Delta^{\{1\}}}$ is homotopic to $g\circ d$.
\end{enumerate}
That is, $s$ makes the two triangles in the following diagram commute up to homotopy:
\[
\xymatrix{
\cdot\ar[r]\ar[d]_f&\cdot\ar[d]^g\\
\cdot\ar[r]\ar@{.>}[ur]|{\ s\ }&\cdot
}
\]
\end{definition}
\begin{definition}[orthogonality]
We say that $f$ \emph{left orthogonal} to $g$ (while $g$ is \emph{right orthogonal} to $f$), in symbols $f\perp g$ if, for any $q\in \iC^{\Delta^{1}}(f,g)$, the space of fillers for $q$ is contractible. 
\end{definition}
\begin{definition}[orthogonality of a class]
Given a subclass $\mathcal{X}\subseteq \iC_1$ we let
\begin{gather*}
{}^{\perp}\mathcal{X}\coloneqq \{y\in\iC_1 \mid y\perp X\}\\
\mathcal{X}^{\perp}\coloneqq \{y\in\iC_1 \mid X\perp y\}.
\end{gather*}
A pair $\F=(\E,\M)$ of sub-classes of $\iC_1$ is a \textsc{pfs} provided ${}^{\perp}\M=\E$ and $\E^\perp=\M$. Furthermore, $\F$ is a \textsc{fs} if, for any $f\in \iC_1$, there exist $e\in \E$ and $m\in\M$ such that $f$ is homotopic to $m\circ e$. 
\end{definition}
Now that we have \textsc{fs}s defined in the setting of $\infty$-categories, let us describe their relation to {\dfs}s. For this we will need to associate to our $\infty$-category $\iC$ a derivator $\D_\iC$ of shape $\Dia$, for a suitable category of diagrams $\Dia$. Since we have taken $\iC$ to be finitely bicomplete, the canonical choice for $\Dia$ is the $2$-category of finite directed categories:
\begin{definition}
A category $I$ is a \emph{finite directed category} if it has a finite number of objects and morphisms, and if there are no directed cycles in the quiver whose vertices are the objects of $I$ and the arrows are the non-identity morphisms in $I$. Equivalently, the nerve of $I$  has a finite number of non-degenerate simplices.\\
We denote by $\fincat\subseteq \cat$ the $2$-category of diagrams spanned by the finite directed categories.
\end{definition}
\begin{proposition}
\cite[Remark \textbf{5.3.10}]{riehl2017kan}
Given an $\infty$-category $\iC$, the composition
$$\D_\iC\colon\xymatrix@R=0pt{
\cat^\opp\ar[r]^{N}&\cat_{\infty}^\opp\ar[r]^{\iC^{(-)}}&\Cat_{\infty}\ar[r]^{\ho}&\Cat\\
I\ar@{|->}[r]&N(I)\ar@{|->}[r]&\iC^{N(I)}\ar@{|->}[r]&\ho(\iC^{N(I)})}$$
is a pre-derivator. If $\Dia$ is a category of diagrams such that $\iC$ has all limits and colimits of shape $\Dia$, then $\D_{\iC}\restriction_{\Dia^\opp}$  is a strong derivator and it is pointed if and only if $\iC$ is pointed. If $\iC$ is stable then, by definition, $\iC$ has all limits and colimits of shape $\fincat$, so $\D_\iC\restriction_{\fincat^\opp}$ is a derivator which, moreover, is stable. 
\end{proposition}

In the rest of this subsection we will work in the following setting: we let $\iC$ be an $\infty$-category, and $\D_\iC\colon \fincat^\opp\to \Cat$ be the associated pre-derivator. Let $\F=(\E,\M)$ be a \textsc{fs} in $\iC$. By \cite[§\textbf{24.10}]{joyal2008notes}, for any $I\in \Dia$ we can define a \textsc{fs} $\F_I=(\E_I,\M_I)$ in $\iC^{N(I)}$, point-wise induced by $\F$. Letting $\mathbb E(I) \coloneqq \ho(\E_I)\subseteq \D^\due_\iC(I)$ and $\mathbb M(I) \coloneqq \ho(\M_I)\subseteq \D^\due_\iC(I)$, one can prove that $\mathbb E$ and $\mathbb M$ are sub-$2$-functors of $\D_\iC$ and that the pair $\F_{\D} \coloneqq (\mathbb E,\mathbb M)$ is a \dpfs on $\D_\iC$. In fact, the unique delicate part is to show that the orthogonality of 1-simplices in $\iC$ implies orthogonality of the corresponding objects in $\D_{\iC}(\due)$. One can do that by hand or using \cite[Lem. \textbf{5.2.8.22}]{HTT}. 
Thus we have constructed a map 
\begin{equation}\label{the_map}
\xymatrix@R=0pt{
\left\{\text{\textsc{fs} in $\iC$}\right\}\ar@{->}[rr]&&\left\{\text{\dfs in $\D_\iC$}\right\}\\
(\E,\M)\ar@{|->}[rr]&&(\E,\M)_\D.
}
\end{equation}
What we would like to understand is whether or not any \dfs in $\D_\iC$ arises from a \textsc{fs} in $\iC$. 
%There is one first natural restriction given by the fact that the {\dfs}s that lie in the image of the above map are all determined by their image in $\D_\iC(\uno)$ (\ie if $(\E,\M)$ is a \textsc{fs} in $\iC$ and $(\E,\M)_\D=(\mathbb E,\mathbb M)$, then $X\in \mathbb E(I)$ if and only if $X_i\in \mathbb E(\uno)$ for all $i\in I$).
%
%\begin{definition}[maximal subderivator]\label{maximal-sub}
%Given a pre-derivator $\D\colon \Dia^\opp\to \Cat$ a sub-2-functor $\X\subseteq \D$ is said to be \emph{maximal} provided 
%\[
%X\in \X(I)\ \Leftrightarrow \ (X_i\in \X(\uno),\ \forall i\in I).
%\]
%This terminology is justified by the fact that, since $\X$ is a a sub-2-functor of $\D$, given $X\in \X(I)$ we always have that $X_i\in \X(\uno)$; thus $\X(I)$ is the biggest possible subclass of $\D(I)$ that can be a component of a sub-2-functor whose component on $\D(\uno)$ is $\X(\uno)$.
%
%A \dfs $\F=(\mathbb E,\mathbb M)$ on $\D$ is said to be \emph{maximal} provided $\mathbb E$ is a maximal sub-2-functor of $\D$. 
%\end{definition}
%
%We can now state properly the main question we are interested in this subsection:
%
%\begin{question}\label{question_infty}
%Is it true that, for any $\infty$-category $\iC$, the map in (\refbf{the_map}) induces a bijection between the \textsc{fs}s in $\iC$ and the maximal {\dfs}s in the associated pre-derivator $\D_\iC$?
%\end{question}
When $\iC=N(\A)$ is the nerve of a 1-category $\A$, $\D_\iC$ is the derivator represented by $\ho(\iC)\cong\A$, so both the \textsc{fs}s on $\iC$ and the  {\dfs}s on $\D_\iC$ correspond bijectively with the (classical) factorization systems on $\A$. 

For the rest of this subsection we concentrate our efforts to find a similar bijection when $\iC$ is stable. We start with the following lemma:

\begin{lemma}\label{lemma_infinity_orth}
Let $\iC$ be a stable $\infty$-category and let $\F=(\mathbb E,\mathbb M)$ be a \dfs on $\D_\iC$. Let $\mathbf{E}$ and $\mathbf M$ be the simplicial subsets of $\iC^{\Delta^1}$ spanned by $\mathbb E(\uno)$ and $\mathbb M(\uno)\subseteq \iC_1$, respectively. Given $X\colon X_0\to X_1$ in $\iC^{\Delta^1}$, consider the following squares in $\iC^{\Delta^1\times \Delta^1}$
\[
p:\qquad\begin{matrix}\xymatrix{
X_0\ar[r]^e\ar[d]_X&C\ar[d]^m\\
X_1\ar@{=}[r]&X_1
}\end{matrix}
\qquad
\text{and}
\qquad
q:\qquad\begin{matrix}\xymatrix{
X_0\ar@{=}[r]\ar[d]_e&X_0\ar[d]^X\\
C\ar[r]_{m}&X_1
}\end{matrix}
\]
If $e\in \mathbf E_0$ and $m\in \mathbf M_0$, then $p$ is an $\mathbf M$-localization and $q$ is an $\mathbf E$-co-localization of $X$ (in the sense of \cite[\adef\textbf{5.2.7.6}]{HTT}). In particular, $\mathbf M$ is reflective, $\mathbf E$ is coreflective in $\iC^{\Delta^1}$ and any reflection of $X$ in $\mathbf M$ (resp., any co-reflection of $X$ in $\mathbf E$) is equivalent to $p$ (resp., $q$).
\end{lemma}
\begin{proof}
By definition of $\mathbf E$-co-localization, we should verify that $q$ induces, for any $e'\in \mathbf E_0$, a weak homotopy equivalence
$\mathrm{Map}_\iC(e',e) \to \mathrm{Map}_\iC(e',X)$. According to Whitehead's theorem, we need to show that for every $k\leq 0$, the map
\[
\mathrm{Ext_{\iC^{\Delta^1}}^k}(e',e)=\D_\iC(\due)(\Sigma^{-k}e',e)\to \D_\iC(\due)(\Sigma^{-k}e',X)= \mathrm{Ext_{\iC^{\Delta^1}}^k}(e',X)
\] 
is an isomorphism of abelian groups. We prove first the case $k=0$. Indeed, fix a quasi-inverse $\Phi_\F\colon \D_\iC(\due)\to \D_\F(\uno)$ to the functor $\Psi_\F\colon \D_\F(\uno)\to \D_\iC(\due)$ (see \athm\refbf{stable_equv_orth}), then 
\begin{align*}
\D_\iC(\due)(e',X)&\cong \D_\iC(\tre)(\Phi_\F(e'),\Phi_\F(X))\\
&\cong \D_\iC(\tre)((0,1)_!e',\Phi_\F(X))\\
&\cong \D_\iC(\due)(e',(0,1)^*\Phi_\F(X))\\
&\cong \D_\iC(\due)(e',e).
\end{align*}
For $k\leq -1$, complete $q$ to a fiber sequence
\[
\begin{matrix}\xymatrix{
X_0\ar@{=}[r]\ar[d]_e&X_0\ar[d]^X\ar[r]&0\ar[d]\\
c\ar[r]_{m}&X_1\ar[r]&C(m)
}\end{matrix}
\]
and, passing to the associated long exact sequence, we are reduced to prove that, for any $k\leq -1$,
\[
0=\D_\iC(\due)\left(\Sigma^{-k}e',1_!C(m)\right)\cong \D_\iC(\uno)\left(\Sigma^{-k}C(e'),C(m)\right).
\]
Since $\Sigma^{-k}e', \Sigma^{-k+1}e'\in \mathbf E_0$ for any $k\leq -1$,  it is enough to prove that 
\[
\D_\iC(\uno)(C(e''),C(m))=0
\]
for any $e''$ such that $e''$ and $\Sigma e''\in \mathbf E_0$. But this follows by the two conditions $e''\horth m$ and $\Sigma e''\horth m$ in the triangulated category $\D_\iC(\uno)$ (they imply that the map $\D_\iC(\uno)(C(e''),C(m))\to \D_\iC(\uno)(e_1,\Sigma m_0)$ is both injective and trivial).
\end{proof}

Using the above lemma, we can now mimic part of the proof of \cite[\aprop \textbf{5.2.8.17}]{HTT} to verify that a \dfs on $\D_\iC$ induces a \textsc{fs} on the original $\infty$-category $\iC$. 

\begin{theorem}\label{infinity_FS_vs_CFS}
Let $\iC$ be an $\infty$-category which is either stable or the nerve of a 1-category. Denote by $\D_\iC\colon \fincat^\opp\to \Cat$ the induced pre-derivator. Then there is a bijective correspondence
\[
\xymatrix{
\left\{\text{\textsc{fs}s in $\iC$}\right\}\ar@{<->}[rr]&&\left\{\text{{\dfs}s in $\D_\iC$}\right\}.
}
\]
\end{theorem}
\begin{proof}
We have already mentioned that, if $\iC$ is the nerve of a 1-category $\A$, then both the \textsc{fs}s on $\iC$ and the {\dfs}s on $\D_\iC$ correspond bijectively with the classical factorization systems on $\A$. Hence, we concentrate on the case when $\iC$ is stable, so that $\D_\iC$ is a stable derivator. Given a \dfs $\F=(\mathbb E,\mathbb M)$ on $\D_\iC$ and letting $(\E,\M)$ be the two classes of 1-simplices in $\iC$ corresponding respectively to $\mathbb E(\uno)$ and $\mathbb M(\uno)$, we have to show that $(\E,\M)$ is a \textsc{fs} in $\iC$. Our strategy will be the following: we faithfully repeat the argument of the proof of the implication (1)$\Rightarrow$(2) in \cite[\aprop \textbf{5.2.8.17}]{HTT} to show that the restriction map 
\[
p\colon \Fun_{\E|\M}(\Delta^2,\iC)\to \Fun(\Delta^{\{0,2\}},\iC)
\] 
is a trivial Kan fibration, where $\Fun_{\E|\M}(\Delta^2,\iC)$ denotes the full subcategory of $\Fun(\Delta^2,\iC)$ spanned by those diagrams 
corresponding to a composition of an element in $\E$ followed by an element in $\M$. By \cite[\emph{ibi}]{HTT}, the fact that $p$ is a trivial Kan fibration is equivalent to say that $(\E,\M)$ is a \textsc{fs}, thus concluding the proof.

Let us start observing that $p$ is a categorical fibration, so it suffices to verify that it is a  categorical equivalence. We do this in two steps. First we let $\sD$ be the full subcategory of $\Fun(\Delta^1\times \Delta^1,\iC)$ spanned by those diagrams of the form
\[\xymatrix{
X\ar[r]^e\ar[d]_f&Y\ar[d]^m\\
Z\ar[r]^g&Z'
}\]
with $e\in \E$, $m\in \M$ and $g$ an equivalence. The map $p$ factors as a composition
\[
\xymatrix{
\Fun_{\E|\M}(\Delta^2,\iC)\ar[r]^(.7){p'}&\sD\ar[r]^(.3){p''}& \Fun(\Delta^{1},\iC)
}
\] 
where $p'$ carries a diagram
\[
\xymatrix{
X\ar[rr]^f\ar[rd]_e&&Z\\
&Y\ar[ur]_m}
\]
to the partially degenerate square
\[\xymatrix{
X\ar[dr]|f\ar[r]^e\ar[d]_f&Y\ar[d]^m\\
Z\ar@{=}[r]&Z
}\]
and $p''$ is given by restriction to the left vertical edge of the diagram. To complete the proof, it will suffice to show that $p'$ and $p''$ are categorical equivalences. For $p'$, this is a general fact proved in \cite[\emph{ibi}]{HTT} (which does not depend on the properties of the pair $(\E,\M)$) so it remains only to show that $p''$ is a trivial Kan fibration.

Let $\T$ denote the full subcategory of $\Fun(\Delta^1,\iC)\times \Delta^1$ spanned by those pairs $(m, i)$ where either $i = 0$ or $m\in \M$. 

The projection map $r \colon \T\to \Delta^1$ is the cartesian fibration associated to the inclusion 
$\Fun_{|\M}(\Delta^1,\iC) \subseteq \Fun(\Delta^1,\iC)$, where $\Fun_{|\M}(\Delta^1,\iC)$ is the full subcategory spanned by the elements of $\M$.
Using Lemma \refbf{lemma_infinity_orth}, we conclude that $r$ is also a co-Cartesian fibration. Moreover, we can identify
\[\sD \subseteq \Fun(\Delta^1\times \Delta^1, \iC) \cong \mathrm{Map}_{\Delta^1} (\Delta^1, \T)\]
with the full subcategory spanned by the co-Cartesian sections of $r$. In terms of this identification, $p''$ is given by evaluation at the initial vertex $\{0\}\subseteq \Delta^1$ and is therefore a trivial Kan fibration, as desired.

The fact that this correspondence really gives a bijection is a consequence of Corollary \ref{everything_is_maximal}.
\end{proof}

%A consequence of the above lemma is that, for $\iC$ a stable $\infty$-category and $\F=(\mathbb E,\mathbb M)$ a \dfs on $\D_\C$ as above, the classes of 1-simplices $(\mathbb E(\uno),\mathbb M(\uno))$ give a \textsc{fs} in $\C$. In fact, the difficult part in proving that $(\mathbb E(\uno),\mathbb M(\uno))$ is a \textsc{fs} is to verify that, given $e\in \mathbb E(\uno)$ and $m\in \mathbb M(\uno)$, then $e\perp m$, that is, that there is a homotopy pullback square
%\[
%\xymatrix{
%\mathrm{Map}_\C(e_1,m_0)\ar[r]\ar[d]&\mathrm{Map}_\C(e_1,m_1)\ar[d]\\
%\mathrm{Map}_\C(e_0,m_0)\ar[r]&\mathrm{Map}_\C(e_0,m_1)
%}
%\]
%Of course, the pullback $\mathrm{Map}_\C(e_0,m_0)\times_{\mathrm{Map}_\C(e_0,m_1)} \mathrm{Map}_\C(e_1,m_1)$ is  $\mathrm{Map}_{\C^{\Delta^1}}(e,m)$, so we are reduced to prove that the induced map 
%\[
%\mathrm{Map}_\C(e_1,m_0)\to \mathrm{Map}_{\C^{\Delta^1}}(e,m)
%\]
%is a weak homotopy equivalence. Consider a factorization of $m$ 
%\[
%\xymatrix@R=10pt{
%m_0\ar[drr]_{e'}\ar[rrrr]^{m}&&&&m_1\\
%&&c\ar[urr]_{m'}
%}
%\]
%where $e'\in\mathbb E(\uno)$ and $m'\in \mathbb M(\uno)$. By the usual cancellation properties, $e'\in \mathbb E(\uno)\cap \mathbb M(\uno)$, so it is an equivalence. By the above lemma,
%\[
%\mathrm{Map}_{\C^{\Delta^1}}(e,m)\cong \mathrm{Map}_{\C^{\Delta^1}}(e,e')\cong \mathrm{Map}_{\C}(e_1,e'_0)\overset{(*)}{\cong} \mathrm{Map}_{\C}(e_1,e'_1)=\mathrm{Map}_{\C}(e_1,m_0),
%\]
%where the equivalence marked by ($*$) holds since $e'$ is an equivalence, giving us that $e\perp m$. Hence, we have proved the following theorem:

%\begin{proof}
%Let $\F=(\E,\M)$ be a \textsc{fs} in $\C$. For any $I\in \Dia$, there is a point-wise induced \textsc{fs} $\F_I=(\E_I,\M_I)$ in $\C^{N(I)}$, see \cite[§\textbf{24.10}]{joyal2008notes}. For any $I\in \Dia$, let $\mathbb E(I) \coloneqq \ho(\E_I)\subseteq \D^\due_\C(I)$ and, similarly, $\mathbb M(I) \coloneqq \ho(\M_I)\subseteq \D^\due_\C(I)$. It is not difficult to show that, $\mathbb E$ and $\mathbb M$ are maximal sub-$2$-functors of $\D$ and, by Lemma \refbf{lemma_infinity_orth}, the pair $\F_{\D} \coloneqq (\mathbb E,\mathbb M)$ is a \dpfs on $\D_\C$. Hence we have constructed a map 
%$$\Phi\colon\xymatrix@R=0pt{
%\left\{\text{\textsc{fs} in $\C$}\right\}\ar@{->}[rr]&&\left\{\text{maximal \dfs in $\D_\C$}\right\}\\
%\F\ar@{|->}[rr]&&\F_\D.
%}$$
%To show that this map is a bijection it is enough to construct an inverse. For this, remember that, by definition, the objects of $\D_\C(\due)$ are exactly the elements of $(\C^{\Delta^{1}})_0=\C_1$ so, given a \dfs $\F=(\mathbb E,\mathbb M)$ on $\D_{\C}$, we can just define $\E \coloneqq \Ob(\mathbb E(\uno))\subseteq \Ob(\D_\C(\due))=\C_1$ and, similarly,  $\M \coloneqq \Ob(\mathbb M(\uno))\subseteq \Ob(\D_\C(\due))=\C_1$. By Lemma \refbf{lemma_infinity_orth}, $\F_\C \coloneqq (\E,\M)$ is  a \textsc{pfs} and, if $\F$ is a  \dfs then $\F_\C$ is a \textsc{fs}. It is not difficult to show that the following map is the inverse of $\Phi$
%\[
%\Psi\colon
%\xymatrix@R=0pt{
%\left\{\text{maximal \dfs in $\D_\C$}\right\}\ar@{->}[rr]&&\left\{\text{\textsc{fs} in $\C$}\right\}\\
%\F\ar@{|->}[rr]&&\F_\C.
%}
%\]
%\end{proof}